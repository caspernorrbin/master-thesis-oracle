Performance, memory usage, and fragmentation of the allocators are critical metrics that were evaluated in this study. These are important factors for a memory allocator, and improvements in one often affect others. Performance was measured by the time required for allocation and deallocation requests. Memory usage refers to the additional memory utilized by the allocator for its operations, and fragmentation refers to the inefficiencies in memory utilization resulting from the allocation strategy.

As previously stated, a significant constraint of this project was the non-integration of the allocator into ZGC, despite this being its intended purpose. This integration would have provided the simplest and most intuitive method of testing the allocator. The challenge of not doing this was to identify other benchmarks that could indicate the allocator's performance within ZGC.

\subsection{Allocator Configurations}
The three versions of the buddy allocator---the binary buddy allocator, the binary tree allocator, and the iBuddy allocator---were tested and compared against each other. They all share the same base configuration of block sizes and regions, with a minimum block size of $16$ bytes, a maximum block size of $256$ KiB, and a total of $8$ regions.

\subsection{Performance}
Performance is influenced by both the type of allocator used and the context of its use. Different allocators perform differently depending on the task; therefore, their performance varies across different programs. Instead of relying on a single program to compare all allocators, primitive measurements were used to show how the allocators performed in allocating and deallocating various sizes and numbers of blocks.

\subsubsection{Measuring Performance}
The time required for a single allocation or deallocation for each block size was observed to provide insight into the performance of each allocator. Additionally, timing the allocation of a constant memory size using various block sizes demonstrated the performance of repeated allocations and deallocations.

Measurements were be taken for every possible block size allocation, ranging from $2^4$ to $2^{18}$. These figures match the potential block sizes in a ZGC small page. To minimize noise, the average of $100 000$ allocations was calculated for each size and each allocator. The duration was recorded before and after the allocation call using the POSIX function \texttt{clock\_gettime()} with the \texttt{CLOCK\_MONOTONIC\_RAW} clock.

Another test involved allocating a larger memory block. This test filled a contiguous 2 MiB memory block for every possible block size ranging from $2^4$ to $2^{18}$, matching ZGC's allocation sizes and page size. For each allocator and block size, the memory was filled $10 000$ times. The duration was recorded using the POSIX function \texttt{clock\_gettime()} with the \texttt{CLOCK\_MONOTONIC\_RAW} clock, starting before the initial allocation and ending after the final one.

These tests showed different performance aspects of the different allocators for various allocation sizes, which were used to draw several conclusions. To estimate the performance of a particular program, one would examine its allocation distribution and compare it with the test results to determine the most suitable allocator for that specific application.

\subsubsection{System Specifications}
Performance evaluations were performed using two different machines, detailed in Table~\ref{table:performancespecs}. Each memory allocator was compiled using GCC 11.4.0, adhering to the C++14 standard (\texttt{-std=c++14}) and using optimization level two (\texttt{-O2}).

\begin{table}[h]
    \begin{tabular}{lll}
        \textbf{Configuration}    & \textbf{Machine A}                                       & \textbf{Machine B}   \\ \hline
        CPU                       & Intel® Core™ i7-1270P                                    & AMD Opteron™ 6282 SE \\ \hline
        Sockets / Cores / Threads & 1 / 4P 8E / 16                                           & 2 / 16 / 32          \\ \hline
        Frequency (Base / Turbo)  & 2.2 GHz / 4.8 GHz                                        & 2.6 GHz / 3.3 GHz    \\ \hline
        L1 Cache                  & 448 KiB                                                  & 512 KiB              \\ \hline
        L2 Cache                  & 9 MiB                                                    & 32 MiB               \\ \hline
        L3 Cache                  & 18 MiB                                                   & 24 MiB               \\ \hline
        Memory                    & 16 GiB                                                   & 126 GiB              \\ \hline
        OS                        & \multicolumn{2}{l}{Ubuntu 22.04.4 LTS (Jammy Jellyfish)}                        \\ \hline
        Kernel                    & 6.5.0-17-generic                                         & 5.15.0-101-generic   \\
    \end{tabular}
    \centering
    \caption{Machines used for performance benchmarks.}
    \label{table:performancespecs}
\end{table}

\subsection{Fragmentation} \label{sec:frageval}
\subsubsection{Internal Fragmentation}
Internal fragmentation occurs when allocation sizes are rounded to the nearest power of two to align with block sizes, and it is identical in all the allocators. This results in unused memory for every allocation that is not a perfect power of two. To quantify this, two counters were used: one for the requested allocation size and another for the total allocated size. Each allocation operation incremented these counters, while deallocation operations decremented them. At any point during program execution, these two counters could be compared to measure the amount of internal fragmentation. The percentage of internal fragmentation, or wasted memory, was calculated as:

$\text{Internal fragmentation} = \frac{\text{Used memory - Requested memory}}{\text{Used memory}}$

\subsubsection{External Fragmentation}
External fragmentation occurs when allocated blocks are placed in memory in a way that inhibits larger allocations. Even if the total space available is greater than the requested size, there may be no single contiguous block large enough to satisfy that request. The different allocators have different policies about where blocks are placed, so fragmentation differs between them.

\subsubsection{Measuring Fragmentation}
Fragmentation is highly dependent on a specific pattern of allocation and deallocation. A program that consecutively only allocates powers of two would experience no internal or external fragmentation. However, this scenario is unrealistic, as programs typically mix allocations and deallocations across a broad range of sizes. To illustrate potential fragmentation levels in a program, a simulation was performed that used the modified allocators in a way that created high fragmentation. This simulation provided a rough estimate of the maximum fragmentation that could arise from typical use of the allocator.

This test used two modules: one for creating a distribution of allocation sizes and another for producing a sequence of allocations and deallocations based on that distribution. Using these, it was possible to measure both internal and external fragmentation at any point within the sequence of allocations.

The chosen allocation distribution was based on the observation that most allocations in Java programs are small, with a few large allocations and occasionally very large ones. Samples were drawn from a Poisson distribution with a mean of $\lambda = 6$. The samples were converted to allocation sizes by using them as exponents with base 2, and a variance of $\pm 50\%$ was applied to the converted samples to introduce spread.

To make the evaluation targeted towards ZGC, the values were aligned to $8$, and any values smaller than $2^4$ or larger than $2^{18}$ were excluded to adhere to the limited allocation size range for small pages. This resulted in a distribution centered on $2^6$, with most values clustering near this point and a gradual but steep decrease in the frequency of higher values.

The sequence of allocations and deallocations was created to introduce significant fragmentation without using a deliberate pattern. First, a sequence based on the allocation size distribution was generated. Following this sequence, allocations were made until a failure occured. Subsequently, half of the allocations by size were randomly deallocated, resulting in fragmentation throughout the entire memory. This process was iterated numerous times to ensure extensive fragmentation.

External fragmentation was measured each time an allocation attempt was unsuccessful, providing insight into the allocator’s ability to manage fragmented memory. Internal fragmentation was measured simultaneously, capturing the impact of random size variations on memory utilization. This approach ensured a thorough evaluation of the allocator's performance under conditions of high fragmentation.

%%% Local Variables:
%%% mode: latex
%%% TeX-master: "main"
%%% End:
