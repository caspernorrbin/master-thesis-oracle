Given the large number of existing allocator designs, one might conclude that most problems with dynamic memory allocation have already been solved. In most cases, a general-purpose allocator designed for use in any system or environment performs well on average in terms of response time and fragmentation. However, there may be areas for improvement depending on how much the use case can be narrowed down. For example, a garbage collector has access to more information about the objects being allocated than most other users of an allocator.

\subsection{Memory Allocators}

One of the first memory allocation techniques is the simple buddy allocator, conceived by Knowlton \cite{buddy}. The buddy allocator partitions memory into blocks, each sized as a power of two. Initially, the entire memory is set as a single block. In the first allocation, this block is divided recursively into halves until the block size closest to the allocation size is obtained. These new differently-sized blocks can then be utilized for subsequent allocations. The two resulting blocks from the splitting of a block are called ``buddies''. When a block is deallocated and its buddy is also unallocated, the two blocks are combined into a larger block. This merging process continues recursively as long as the buddy is also unallocated.

The simple design of the buddy allocator provides ample room for enhancements. Peterson and Norman presented a more generalized version of the buddy allocator~\cite{genbuddy}, allowing block sizes that do not necessarily have to be powers of two and allowing for buddies of unequal sizes. Recent advances have also included making the allocator lock-free~\cite{nbbs} and enabling the same block to exist in multiple sizes simultaneously to improve the efficiency of smaller allocations~\cite{park2014ibuddy}.

The buddy allocator has seen wide use in real environments and applications. A notable example is its use within the Linux kernel for allocating physical memory pages~\cite{linuxbuddy}. In addition, the buddy allocator can be combined with various other allocation techniques to improve efficiency. For example, \texttt{jemalloc}~\cite{jemalloc}, created for the FreeBSD operating system, is a general-purpose allocator that incorporates multiple strategies, including the buddy allocator, to achieve overall efficiency.

\subsection{Free-Lists in Garbage Collection}

Using free-lists in garbage collection systems is not a new concept. For example, the basic mark-sweep algorithm, as outlined in Section~\ref{sec:gc}, necessitates the use of a free-list. The static position of live objects can lead to significant fragmentation. Consequently, simpler allocation methods, such as bump-pointer allocation, become impractical. Instead, employing a strategy based on free-lists for allocation proves to be a more effective approach to managing and allocating memory surrounding live objects.

Among the free-list strategies outlined in Section~\ref{sec:freelist_allocation}, the segregated-fit method is particularly noted for its effectiveness, as emphasized by Jones et al.~\cite{gchandbook}. Programs that anticipate garbage collection generally allocate more than those that manually handle memory. Therefore, the significance of segregated-fit free-lists becomes particularly crucial as program allocations increase.

The Concurrent Mark Sweep (CMS) garbage collector~\cite{cms}, introduced in JDK 1.4, stands as a concrete example of free-lists being used within real-world garbage collectors. CMS implements a concurrent variant of the mark-sweep algorithm, thereby utilizing free-lists for managing memory. Although it was deprecated in JDK 9, CMS demonstrates the practicality of using free-lists in garbage collectors. The presence of free-lists, particularly in the context of Java program execution, highlights their versatility and adaptability across different garbage collection algorithms and implementations.

It is evident that there exists a space for free-list-based allocators to exist within garbage collection systems. Furthermore, it is known that the use of multiple allocation strategies in tandem can enhance allocation efficiency. Therefore, it is reasonable to suggest implementing a free-list-based allocator, such as the buddy allocator, into a garbage collection system, such as ZGC, as an alternative strategy to boost performance in certain situations.

%%% Local Variables:
%%% mode: latex
%%% TeX-master: "main"
%%% End:
