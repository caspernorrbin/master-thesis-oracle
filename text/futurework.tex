\subsection{Integration Into ZGC} \label{sec:futureworkZ}
The most significant future step is to integrate the allocators into ZGC. As highlighted in Section~\ref{sec:individual_contrubitons}, N. Gärds has in parallel to this thesis explored the integration of an allocator into ZGC~\cite{niclas}, similar to the use cases suggested here. Their research focuses on implementing a free-list-based allocator, not necessarily a buddy allocator, to enhance the efficiency of relocating objects within ZGC. Future efforts might involve building on Gärds' research to incorporate a buddy allocator into ZGC, or alternatively, starting from another perspective and integrating the allocators for scenarios not considered by Gärds.

When integrating a buddy allocator specifically, the garbage collector can implement strategies to mitigate some of its drawbacks. The primary issue is the considerable internal fragmentation during allocation. To minimize this, ZGC might allocate multiple objects simultaneously to align more closely with size requirements, or it could use fragmented memory from an allocation for future use. This approach requires that objects allocated together be deallocated collectively. However, as detailed in Section~\ref{sec:freerangeexpl}, when deallocating ranges, this no longer becomes a consideration.

\subsection{Smaller Minimum Allocation Size}
As of now, the minimum allocation size in ZGC is 16 bytes, restricted by the header size of Java objects. Efforts are underway to decrease the size of these headers, lowering the smallest object size to 8 bytes \cite{liliput}. However, since this is still under development, this change was not considered for this thesis.

The primary impact of this change would be the introduction of a new bottom layer to the allocators. Practically, this results in memory overhead per page doubling as it becomes necessary to manage and track smaller allocations. Moreover, for the binary buddy and the iBuddy allocators, storing the two pointers needed for the free-list would no longer be possible. Instead, these pointers would need to be transformed into offsets from the beginning of the page. Although the size of pages makes this a non-issue, this adjustment would still introduce some performance overhead.

\subsection{Further Allocation Optimizations}


Improving the binary tree storage format in the binary tree allocator might be feasible. Currently, the levels of the tree are stored sequentially, requiring progressively larger jumps when iterating the levels. To increase spatial locality, the larger tree could be broken down into several subtrees that can be stored consecutively. This would place the segments closer to each other, reducing the likelihood of cache misses, which may increase performance.

Currently, all the adapted allocators use the same concurrency approach, which involves dividing memory into regions. However, since their designs differ, it would be worth investigating using different designs for each allocator. As an example, a lock-free mechanism could be used to achieve concurrency for the binary tree allocator. As the binary tree allocator navigates its tree structure, compare-and-swap (CAS) atomic operations could replace the current subtraction operations. This change would require increased caution, and the allocator would need to reserve more space during tree traversal so that it can allocate where it expects. This space would be returned after the operation is complete. Conflicts could arise with other threads, but threads can back off to resolve these conflicts. Consequently, keeping the region implementation is advisable to distribute allocations evenly, but with the locking code removed.

% alla samma, dåligt
% undersöka anpassing för specifika versioner
% binary tree extra bra!!! därför att......
% 

