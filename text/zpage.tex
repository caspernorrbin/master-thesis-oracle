
ZGC is a region-based garbage collector which allocates objects inside differently-sized regions of memory that are referred to as pages. When new objects are created, they are allocated inside pages with respect to their size. Pages are classified as one of three different types: \textit{Small}, \textit{Medium} or \textit{Large}, which support allocations of different size ranges, as specified in Table~\ref{table:zpage_sizes}. All allocations are aligned to 8 bytes and the smallest supported allocation size is, at the time of writing, 16 bytes. As a result of all allocations having the same alignment, their offset from the start of the page will also be a multiple of 8 bytes.

\begin{table}[H]
    \centering
    \begin{tabular}{lllll}
        Page Type & Page Size   & \multicolumn{3}{l}{Object Size}      \\ \hline
        Small     & 2 MB        & \multicolumn{3}{l}{{[}16B, 256KB{]}} \\
        Medium    & 32 MB       & \multicolumn{3}{l}{(256KB, 4MB)}     \\
        Large     & $\geq$ 4 MB & \multicolumn{3}{l}{$\geq$ 4MB}       \\
    \end{tabular}
    \caption{Page sizes in ZGC (Taken from Yang and Wrigstad~\cite{zgc:zpage_size_table}).}
    \label{table:zpage_sizes}
\end{table}

Pages in ZGC include metadata that allows them to perform allocations efficiently, as well as ensure safe execution while running concurrently executing threads. An overview of this metadata is shown in Figure~\ref{fig:zpages}, where the most important attributes are the: \textit{Bump Pointer}, \textit{Live Map}, \textit{Sequence Number}, and \textit{Age}, which are explained in detail below.

\begin{description}
    \item[Bump Pointer]
        The bump pointer is a pointer within the page's memory range that stores the location of where new allocations are made inside the page. Its functionality and use-case are explained in detail in Section~\ref{sec:bump_pointer}. All allocations inside pages in ZGC are currently done using bump pointers. In Figure~\ref{fig:zpages}, the bump pointer is displayed below four allocated objects, indicating that the next position to allocate objects at would be below the fourth object.
        
    \item[Live Map]
        The live map is used to keep track of which objects are currently live and have been marked as reachable by the program. Objects not marked in the live map are considered dead. The live map is shown on the right-hand side of Figure~\ref{fig:zpages}. It is constructed during the marking phase of the GC cycle as live objects are encountered during memory traversal and marked in the live map. The live map is represented by a bitmap, where each bit is mapped to an 8-byte chunk of the page. If a bit is set, there is a live object at the corresponding address on the page. The live map is used during the compacting phase of the GC in order to know how much memory is alive inside the page.
        \newpage
    \item[Sequence Number]
        Every page has a sequence number denoting during which GC cycle it was created. If the GC is on its $N$th cycle, pages created during that GC cycle will have a sequence number $N$. Pages with sequence number $N$ are called \textit{allocating} pages, and are the only pages which allocate new objects for mutator threads. Pages with a sequence number below the current GC cycle, $0$--$N-1$ are called \textit{relocatable} pages since there will be no additional allocations done inside them during the GC cycle, and garbage can be reclaimed with the use of relocation. For example, if the current Java program has performed 5 GC cycles, any allocations after that will be exclusively done on pages with sequence number 5. If the program decides to run a 6th cycle, only garbage from pages with sequence numbers 0 through 5 will be collected, and new allocations will be done on pages with sequence number 6.
        
    \item[Age]
        ZGC is a generational garbage collector, meaning it treats objects of varying ages differently to improve performance. This is based on the observation that most objects survive for only a short period and those that survive for longer tend to live for a long time. This is known as the weak generational hypothesis. ZGC applies this approach on a page-by-page basis, where all allocations on a page share the same age, which is conveniently stored on each page instead of for every object. Objects are grouped into three different age categories: \textit{Eden}, \textit{Survivor} and \textit{Old}. The \textit{Eden} age is for first-time allocation, \textit{Survivor} signifies objects that have survived one or more GC cycles and \textit{Old} those that have survived a threshold of cycles in the \textit{Survivor} age. Classifying pages this way allows the GC to treat pages of a certain age differently, which is especially useful for \textit{Eden} pages, which tend to accumulate garbage quickly.
\end{description}

\begin{figure}[H]
    \centering
    \includesvg[scale=0.8]{figures/zpage_withage.svg}
    \caption{Illustration of a page in ZGC showing what kind of metadata is kept track of to facilitate allocation and object bookkeeping. The page contains three live objects marked in green and with the corresponding bit set in the live map to the right. Red objects are previously allocated objects which have since become unreachable garbage memory.}
    \label{fig:zpages}
\end{figure}

\subsubsection{The Garbage Collection Cycle}

Figure~\ref{fig:zgc_timeline} shows a timeline of a garbage collection cycle in ZGC, made up of an example heap and pages. The timeline shows the different phases of a cycle and how the garbage collector prepares for relocating memory to free unused memory. The three different steps in the timeline show:

\vspace*{-0.4cm}

\begin{enumerate}[label=\alph*)]
    \item The initial state of the heap before the GC cycle starts. The left page has about 30\% free memory and the right page has about 50\%. The current GC cycle is 1, which tells us that pages with sequence number 1 are allocating objects.
    \item The GC cycle has started, and the old pages have been locked. New allocations are now done on new pages instead of the old ones. The old pages are now guaranteed to not allocate any new objects, which makes it possible for the marking phase to begin to traverse the live objects.
    \item The GC has finished the marking phase and now has liveness data that indicate where live objects are allocated inside the relocatable pages.
\end{enumerate}

\vspace*{-0.4cm}

\begin{figure}[H]
    \centering
    \begin{subfigure}[t]{.214\textwidth}
        \centering
        \includesvg[width=1\textwidth]{figures/zrel1.svg}
        \caption{Initial state of the heap before the first GC cycle. The gray color indicates the page is allocating objects.}
        \label{fig:zrel1}
    \end{subfigure}
    \hfill\vline\hfill
    \begin{subfigure}[t]{.32\textwidth}
        \centering
        \includesvg[width=1\textwidth]{figures/zrel2.svg}
        \caption{The state of the heap after marking has started. The blue color indicates that the page is no longer being allocated on.}
        \label{fig:zrel2}
    \end{subfigure}
    \hfill\vline\hfill
    \begin{subfigure}[t]{.32\textwidth}
        \centering
        \includesvg[width=1\textwidth]{figures/zrel3.svg}
        \caption{The state of the heap after all live objects are marked. The red and green colors indicate that the marking phase has finished, and the pages have liveness data that explain where live objects(green) are located}.
        \label{fig:zrel3}
    \end{subfigure}
    \caption{Timeline of a garbage collection cycle, showing the state of the heap in the different phases.}
    \label{fig:zgc_timeline}
\end{figure}

\vspace*{-0.49cm}

ZGC frees memory marked as garbage by copying live objects from one page to another and re-purposing the source page for new allocations. This leads to the fragmented layout of allocated memory to be compacted which makes it possible to fit all live objects on multiple pages onto a single one. The process of copying live objects is also referred to as relocation. When performing relocation, two different scenarios can occur. Either there is enough free memory available in some page where we can allocate, or the heap is full. If the heap is full, objects need to be compacted onto the same page that they are already located on, which is called in-place compaction. The first scenario is illustrated in Figure~\ref{fig:zrel_new}. This requires the heap to have enough free space available to create a new page during the time of relocation. The GC can reclaim garbage memory by relocating live objects from sparsely populated pages into a more dense configuration inside a new page. The fragmentation and also total heap usage is therefore reduced.

\begin{figure}[H]
    \centering
    \begin{subfigure}[t]{.2\textwidth}
        \centering
        \includesvg[width=1\textwidth,height=1.2885\textwidth]{figures/zrel_new1.svg}
        \caption{A page has been selected for relocation due to high fragmentation.}
        \label{fig:zrel_new1}
    \end{subfigure}
    \hfill\vline\hfill
    \begin{subfigure}[t]{.2\textwidth}
        \centering
        \includesvg[width=1\textwidth]{figures/zrel_new2.svg}
        \caption{A new page is created with a new sequence number that is used as a relocation target.}
        \label{fig:zrel_new2}
    \end{subfigure}
    \hfill\vline\hfill
    \begin{subfigure}[t]{.2\textwidth}
        \centering
        \includesvg[width=1\textwidth]{figures/zrel_new3.svg}
        \caption{The objects from the first page are relocated to the new page.}
        \label{fig:zrel_new3}
    \end{subfigure}
    \hfill\vline\hfill
    \begin{subfigure}[t]{.2\textwidth}
        \centering
        \includesvg[width=1\textwidth]{figures/zrel_new4}
        \caption{All objects have been copied to the new page and the old page is re-purposed.}
        \label{fig:zrel_new3}
    \end{subfigure}
    \caption{An example of how a successful relocation is done when the heap has enough space to allocate a new page.}
    \label{fig:zrel_new}
\end{figure}

Since a garbage collection is started as a reaction to high memory pressure, the available memory might be limited. If a new page cannot be allocated, the first page is ``created'' by compacting its objects in-place, and then continuing by relocating the live objects of other pages onto that page. An in-place compaction is potentially an expensive operation, which requires the thread performing the re-arrangement to write to the memory of the page while other threads may read its contents. To do this safely, any threads trying to read from the page during this process must be paused, which removes concurrent execution of the page. The process of in-place compaction of a page is illustrated in Figure~\ref{fig:zrel_in}.

\begin{figure}[H]
    \centering
    \begin{subfigure}[t]{.2\textwidth}
        \centering
        \includesvg[width=1\textwidth,height=1.2885\textwidth]{figures/zrel_in1.svg}
        \caption{The first live object on the page is moved to the start of the page.}
        \label{fig:zrel_in1}
    \end{subfigure}
    \hfill\vline\hfill
    \begin{subfigure}[t]{.2\textwidth}
        \centering
        \includesvg[width=1\textwidth,height=1.2885\textwidth]{figures/zrel_in2.svg}
        \caption{The second live object is moved as close to the start as possible.}
        \label{fig:zrel_in1}
    \end{subfigure}
    \hfill\vline\hfill
    \begin{subfigure}[t]{.2\textwidth}
        \centering
        \includesvg[width=1\textwidth,height=1.2885\textwidth]{figures/zrel_in3.svg}
        \caption{All objects have been moved, and the space after is made available.}
        \label{fig:zrel_in1}
    \end{subfigure}
    \hfill\vline\hfill
    \begin{subfigure}[t]{.2\textwidth}
        \centering
        \includesvg[width=1\textwidth,height=1.2885\textwidth]{figures/zrel_in4.svg}
        \caption{The bump pointer is moved to the beginning of available space.}
        \label{fig:zrel_in1}
    \end{subfigure}
    \caption{An example of how in-place compaction is done to remove fragmentation.}
    \label{fig:zrel_in}
\end{figure}

%%% Local Variables:
%%% mode: latex
%%% TeX-master: "main"
%%% End:
