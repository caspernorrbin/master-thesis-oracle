The methodology was divided into five distinct phases:

\begin{enumerate}
    \item \textbf{Implement the Reference Design:} Implement the original binary buddy allocator and verify its functionality to establish a foundation for further development.
    \item \textbf{Investigate Key Aspects:} Identify important aspects of using a memory allocator within a garbage collection context through research and experimentation with source code.
    \item \textbf{Make Modifications Based on Existing Research:} Modify the reference allocator based on prior research to enhance its performance.
    \item \textbf{Adapt for Garbage Collection:} Further modify the allocator for specific use cases within garbage collection.
    \item \textbf{Evaluate Performance:} Measure memory fragmentation and allocation speed for both the reference allocator and the modified versions.
\end{enumerate}

The initial phase involved implementing the original binary buddy allocator as envisioned by Knowlton \cite{buddy}. This implementation provided a deeper understanding of its functionality and established a solid foundation for subsequent development. The base allocator was tested using both unit tests and real-world applications. By exporting the allocator as a shared library within a wrapper, it was possible to test real-world programs by overriding the malloc/free operations using the \texttt{LD\_PRELOAD} environment variable.

Once the base allocator was implemented, phase two focused on understanding the nuances of memory allocation within the context of garbage collection and ZGC. This involved conducting a literature review, examining specific parts of the ZGC source code, and performing experiments. This phase was crucial for grasping the complexities of ZGC and its memory allocation mechanisms, identifying potential areas for modification, and determining which changes would have the greatest impact.

With a thorough understanding of the base allocator and ZGC, the third phase involved integrating modifications inspired by prior research. The most significant changes were applied, sometimes requiring the implementation of conflicting concepts, which were then compared and evaluated. The focus was on leveraging existing work to enhance overall performance, rather than to start from scratch. These adjustments were not exclusively tailored for garbage collection but aimed to improve general performance or produce positive results when used in garbage collection processes.

In the fourth phase, distinct improvements were implemented that were specifically relevant to the allocator's use in a garbage collection environment. These modifications were derived from insights and knowledge acquired in the preceding phases, rather than based on existing research. The goal was to introduce features tailored for specific scenarios within ZGC without compromising performance in other aspects of ZGC or elsewhere.

The implementation process was incremental, with each minor modification being tested and validated before proceeding to the next step. As the allocator evolved, the number of unit tests increased, which ensured comprehensive test coverage of its capabilities. Additionally, the allocator was tested with preexisting programs after each modification to ensure stability and performance.

The final phase involved evaluating the reference allocator and the modifications. Several key metrics were considered during the evaluation, including fragmentation, memory usage, and allocation speed. These metrics were essential for assessing the allocator's performance and comparing different versions. Certain metrics may hold greater significance in specific scenarios, while others may have different requirements. A detailed explanation of the evaluation process can be found in Section~\ref{sec:evaluation}.

%%% Local Variables:
%%% mode: latex
%%% TeX-master: "main"
%%% End:
