Effective memory management is a crucial part of any software. It can be categorized into either manual memory management, where the developer is responsible for both allocating and deallocating memory, or automatic memory management, where the system handles memory on behalf of the developer. Garbage collection, a type of automatic memory management, identifies and reclaims memory that is no longer in use. Garbage collection is broad, with many possible specific implementations that accomplish the same goal.

Java applications run within a Java Virtual Machine (JVM), and one such example is the Open Java Development Kit (OpenJDK). OpenJDK includes various garbage collectors, including the Z garbage collector (ZGC). ZGC organizes memory into pages that are used concurrently. Objects are allocated sequentially on these pages through a method known as bump-pointer allocation, where a pointer tracks the position of the most recently allocated object, increasing it, or ``bumping'' it, with each new allocation.

Bump-pointer allocation, being sequential, has some significant constraints, namely the inability to reuse spaces from dead objects, which results in increased fragmentation. To resolve this, ZGC either relocates all active objects to a new page, allowing the original page to be reset, or moves all objects one by one to the top of the page. An alternative to this method is employing a free-list-based allocator, which maintains a list of all unoccupied memory, thereby allowing allocations in any available space, independent of prior allocations.

Historically, free-lists have been utilized within garbage collectors. Concurrent Mark Sweep (CMS) is one such example, a garbage collector that relied on free-lists. It was once part of OpenJDK until its deprecation and subsequent removal. CMS could perform allocations without the constraints imposed by bump-pointer allocation methods, thanks to its use of free-lists. The advantages of using free-lists are evident. This thesis investigates the potential integration of a free-list-based allocator within ZGC and examines possible adaptations that can boost allocator efficiency. Using a free-list-based allocator within an existing garbage collector offers significant advantages, as the garbage collector has more information available about the objects it allocates that the allocator can use.

%%% Local Variables:
%%% mode: latex
%%% TeX-master: "main"
%%% End:
