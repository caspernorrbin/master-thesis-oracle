% MUST use a4paper option
% MAY use twoside, smaller font, and other class - but not for Självständigt arbete i IT
\documentclass[a4paper,12pt]{article}
% Use UTF-8 encoding in input files
\usepackage[utf8]{inputenc}
% Use T1 font encoding to make the \hyphenation command work with UTF-8
\usepackage[T1]{fontenc}

% Om ni skriver på svenska, använd denna rad:
% \usepackage[english,swedish]{babel}
% If you are writing in English, use the following line INSTEAD of the previous (note order of parameters):
\usepackage[swedish,english]{babel}

% Use the template for thesis reports
\usepackage{UppsalaExjobb}

\usepackage{enumitem}
\usepackage{svg}
\usepackage{subcaption}
\usepackage{caption}
\usepackage{mwe}

% För att göra ett index behövs
%  - \usepackage{makeidx}
%  - \makeindex i "preamble", dvs före \begin{document}
%  - \printindex, typiskt sist, före \end{document}
% - och att man lägger in \index{ord} på olika ställen i dokumentet
\usepackage{makeidx}
\makeindex


% Designval: per default används styckesindrag, men ibland blir det snyggare/mer lättläst med tomrad mellan stycken. Det åstadkoms av de följande raderna.
% Tycker ni om styckesindrag mera, kommentera bort nästa två rader.
\parskip=0.8em
\parindent=0mm

% Designval: vill ni ha en box runt figurer istället för strecken som är default, av-kommentera raden nedan. Obs att både \floatstyle och \restylefloat behövs.
%\floatstyle{boxed} \restylefloat{figure}

\begin{document}
% För att ställa in parametrar till IEEEtranS/IEEEtranSA behöver detta ligga här (före första \cite).
% Se se IEEEtran/bibtex/IEEEtran_bst_HOWTO.pdf, avsnitt VII, eller sista biten av IEEEtran/bibtex/IEEEexample.bib.
%%%% OBS: här ställer ni t.ex. in hur URLer ska beskrivas.
\bstctlcite{rapport:BSTcontrol}

% Set title, and subtitle if you have one
\title{Allocators in OpenJDK} % och uppsatsmetodik
% Use this if you have a subtitle
%\subtitle{beskrivande men gärna lockande}
%\subtitlesubtitle

% Set author names, separated by "\\ " (don't forget the space, or use newline)
% List authors alphabetically by LAST NAME (unless someone did significantly more/less, which should not be the case)
% For drafts, include your email addresses to make it easier to send peer reviews
\author{Casper Norrbin}

% Visa datum på svenska på förstasidan, även om ni skriver på engelska!
\date{\begin{otherlanguage}{swedish}  %\foreignlanguage doesn't seem to affect \today?
    Juni 2024
  \end{otherlanguage}}

% Använd detta om året för rapporten inte är innevarande år
%\year=2018

% Ange handledare, ämnesgranskare, examinator om dessa finns
\handledare{Erik Österlund}
% There is also \exthandledare
\reviewer{Tobias Wrigstad}
\examinator{Lars-Åke Nordén}

% This creates the title page
\maketitle

% Change to frontmatter style (e.g. roman page numbers)
\frontmatter

%%%% OBS: Läs också källkoden till alla text/X.tex.
%%%% Tips: ni kan använda separata filer för de olika delarna i er rapport på motsvarande sätt,
%%%% men använd inte samma filnamn!

\begin{abstract}
  In the current software development environment, Java remains one of the major languages, powering numerous applications. Central to Java's effectiveness is the Java Virtual Machine (JVM), with HotSpot being a key implementation. Within HotSpot, garbage collection (GC) is critical for efficient memory management, where one collector is Z (ZGC), designed for minimal latency and high throughput. \\

ZGC primarily uses bump-pointer allocation, which, while fast, can lead to fragmentation issues. An alternative allocation strategy involves using free-lists to dynamically manage memory blocks of various sizes, such as the buddy allocator. This thesis explores the adaptation and evaluation of buddy allocators for potential integration within ZGC, aiming to enhance memory allocation efficiency and minimize fragmentation. \\

The thesis investigates the binary buddy allocator, the binary tree buddy allocator, and the inverse buddy allocator, assessing their performance and suitability for ZGC. Although not integrated into ZGC, these exploratory modifications and evaluations provide insight into their behavior and performance in a GC context. The study reveals that while buddy allocators offer promising solutions to fragmentation, they require careful adaptation to handle the unique demands of ZGC. \\

The conclusions drawn from this research highlight the potential of free-list-based allocators to improve memory management in Java applications. These advances could reduce GC-induced latency and enhance the scalability of Java-based systems, addressing the growing demands of modern software applications.

%%% Local Variables:
%%% mode: latex
%%% TeX-master: "main"
%%% End:

\end{abstract}

\begin{sammanfattning}
  Översätts från abstract när den anses vara färdig/korrekt.

%%% Local Variables:
%%% mode: latex
%%% TeX-master: "main"
%%% End:

\end{sammanfattning}

% Innehållsförteckningen här.
\tableofcontents

% Här kan man också ha \listoffigures, \listoftables

\cleardoublepage

% Change to main matter style (arabic page numbers, reset page numbers)
\mainmatter

% Here comes the text of the report.

\section{Introduction}
\label{sec:introduction}
Effective memory management is a crucial for any software system. It can be categorized into either manual memory management, where the developer is responsible for both allocating and deallocating memory, or automatic memory management, where the system handles memory on behalf of the developer. Garbage collection, a type of automatic memory management, identifies and reclaims memory that is no longer in use. There are various different implementations of garbage collection that achieve this goal.

Java applications run within a Java Virtual Machine (JVM), and one such example is the Open Java Development Kit (OpenJDK). OpenJDK includes various garbage collectors, including the Z garbage collector (ZGC). ZGC organizes memory into pages that are used concurrently. Objects are allocated sequentially on these pages through a method known as bump-pointer allocation, where a pointer tracks the position of the most recently allocated object, increasing it, or ``bumping'' it, with each new allocation.

Bump-pointer allocation, being sequential, has some significant constraints, namely the inability to reuse spaces from dead objects, which results in increased fragmentation. To resolve this, ZGC either relocates all active objects to a new page, allowing the original page to be reset, or moves all objects one by one to the top of the page. An alternative to this method is employing a free-list-based allocator, which maintains a list of all unoccupied memory, thereby allowing allocations in any available space, independent of prior allocations.

Historically, free-lists have been utilized by garbage collectors. Concurrent Mark Sweep (CMS) is one such example, a garbage collector that relied on free-lists. It was once part of OpenJDK until its deprecation and subsequent removal. CMS could perform allocations without the constraints imposed by bump-pointer allocation methods, thanks to its use of free-lists. The advantages of using free-lists are evident. This thesis investigates the potential integration of a free-list-based allocator within ZGC and examines possible adaptations that can boost allocator efficiency. Using a free-list-based allocator within an existing garbage collector offers significant advantages, as the garbage collector has more information available about the objects it allocates that the allocator can use.

%%% Local Variables:
%%% mode: latex
%%% TeX-master: "main"
%%% End:


\subsection{Purpose and Goals}
\label{sec:purpose}
The purpose of this thesis is to investigate potential adaptations to an existing free-list-based allocator for integration with ZGC and to implement and evaluate these modifications. The results of these modifications offer insight into the functionality of a free-list-based allocator in ZGC and its potential behavior. The goal is to leverage these insights to direct subsequent efforts in the domain of free-list-based allocators in garbage collection.

The questions this thesis seeks to answer are the following:
\begin{enumerate}
  \item What considerations need to be made when adapting an allocator for use within ZGC?
  \item How can these considerations be used to implement adaptations to a free-list-based allocator to improve its performance and/or memory efficiency?
\end{enumerate}

%%% Local Variables:
%%% mode: latex
%%% TeX-master: "main"
%%% End:


\subsection{Delimitations}
\label{sec:delimitations}
The primary delimitation of this thesis is the exclusion of allocator integration within ZGC. The focus will be on examining the allocators and their adaptations independently. Additionally, the modifications will focus solely on small pages in ZGC, as these are the most common. Consideration of medium and large pages has been excluded from this thesis. However, insights gained from small pages may guide potential adaptations for medium and possibly large pages.

%%% Local Variables:
%%% mode: latex
%%% TeX-master: "main"
%%% End:


\subsection{Individual Contributions}
\label{sec:individual_contrubitons}
The primary author and principal contributor to this thesis is Casper Norrbin. Casper independently developed all sections, except Section~\ref{sec:background}, which was collaboratively authored with Niclas Gärds and Joel Sikström. Both Niclas and Joel pursued their theses in parallel with this one, with Joel focusing on adapting the two-level segregated fit allocator~\cite{joel} and Niclas on integrating an allocator into ZGC~\cite{niclas}. Joel's work resulted in distinct adaptations and outcomes, while Niclas explored the challenges and opportunities of integrating an allocator into ZGC, a natural extension of both this work and Joel's research.

For the section developed in collaboration with Niclas and Casper, the contributions are: Sections~\ref{sec:memory_management},~\ref{sec:fragmentation} and~\ref{sec:memory_allocation} are authored solely by Joel, Sections~\ref{sec:gc} and~\ref{sec:openjdk} are authored solely by Casper, and Section~\ref{sec:zgc} is authored solely by Niclas.

%%% Local Variables:
%%% mode: latex
%%% TeX-master: "main"
%%% End:


\paragraph{Acknowledgement}
\input{text/acknowledgement}

\newpage
\section{Background}
\label{sec:backgrond}
\input{text/background}

\subsection{Memory Management}
\label{sec:memory_management}

Memory management~\cite{gchandbook} is typically categorized as either manual or automatic. Manual memory management involves explicitly managing memory by the programmer, which is commonly used in low-level languages like C and C++. Automatic memory management is handled automatically by the system, without the need for the programmer to intervene. The most common technique for automatic memory management is garbage collection which automatically tracks liveness of memory and reclaimed unused memory. Languages like Python and Java are examples of programming languages that do this.

Memory is most commonly allocated during runtime of the program as opposed to statically allocating everything in advance, during compilation for example. The process of allocating memory during runtime is referred to as dynamic memory management, which presents several challenges for reliably being able to satisfy allocation requests and maintain operational stability over extended periods.

The main challenge with dynamic memory allocation is that an allocation might fail due to memory exhaustion. Exhaustion may arise either due to the program requesting more memory than is available in the system or from the circumstance where free memory is available but cannot be reused. The latter case is sometimes referred to as just wasted memory, but to be precise we will classify it as either internal or external fragmentation~\cite{gchandbook}, which will be discussed further below.

%%% Local Variables:
%%% mode: latex
%%% TeX-master: "main"
%%% End:


\subsection{Fragmentation}
\label{sec:fragmentation}

We classify fragmentation as being either internal or external. Internal fragmentation is considered wasted space due to alignment, which allocates extra memory to meet requirements by the allocator or hardware for example. Figure~\ref{fig:internal_fragmentation} shows an example of when a user has requested 100 bytes, where the allocator has instead allocated 128 bytes to meet the requirements of the allocator. The last 28 bytes are considered wasted space, or internal fragmentation, as the user will not know of its existence and will end up being unused.

\begin{figure}[h]
    \centering
    \vspace*{0.2cm}
    \includesvg[width=0.7\textwidth]{figures/internal_fragmentation.svg}
    \vspace*{0.2cm}
    \caption{A memory region containing one allocated piece of memory that is 128 bytes large. However, the user only requires 100 bytes and thus, 28 bytes are wasted.}
    \label{fig:internal_fragmentation}
\end{figure}

External fragmentation occurs when there is enough memory in total available to satisfy a request, but the request nevertheless cannot be satisfied because the available memory is dispersed in several chunks, none of which is large enough to satisfy the request. This is illustrated in Figure~\ref{fig:external_fragmentation}, where a total of 80 bytes is available but dispersed across the memory region. Consequently, the largest allocation that the system can accommodate is 32 bytes, as the single largest contiguous chunk of memory is of this size.

\begin{figure}[h]
    \centering
    \vspace*{0.2cm}
    \hspace*{1.2cm}
    \includesvg[width=0.9\textwidth]{figures/external_fragmentation.svg}
    \vspace*{0.2cm}
    \caption{A memory region containing several allocated blocks with unused space between them. Although the total sum of the unused portions is 80 bytes, a single request of more than 32 bytes cannot be fulfilled.}
    \label{fig:external_fragmentation}
\end{figure}

Effectively managing and reducing fragmentation is crucial for optimizing memory usage in long-running programs~\cite{gchandbook}. For example, if memory becomes too fragmented, the system may not be able to satisfy allocation requests and be forced to either collect garbage sooner than necessary or terminate if not using a garbage collector~\cite{gchandbook}.

%%% Local Variables:
%%% mode: latex
%%% TeX-master: "main"
%%% End:


\subsection{Memory Allocation}
\label{sec:memory_allocation}

In this section, we describe two fundamental memory allocation strategies, called sequential allocation and free-list allocation~\cite{gchandbook}. Additionally, we discuss the more complex case of using multiple free-lists, called segregated-fits.

\subsubsection{Sequential Allocation}
\label{sec:seq_allocation}
\label{sec:bump_pointer}

Sequential allocation is a method used for allocating memory within a contiguous chunk of memory. In this approach, a pointer is used to track the current position within the memory chunk. As new objects are allocated, the pointer is advanced by the size of the object, along with extra memory that might be needed for alignment purposes. For this reason, sequential allocation is also known as bump-pointer allocation due to the incremental ``bumping'' of the pointer with each new allocation. 

This approach is simple and efficient and is highly effective in situations where memory fragmentation is not a significant concern and where a predictable, sequential layout is desirable. However, it may not be the most suitable choice for all scenarios, especially where memory is not available in a large contiguous chunk. In that case, a more sophisticated memory management strategy might be required. 

\subsubsection{Free-List Allocation}

An alternative to sequential allocation is free-list allocation, which involves maintaining a record of the location and size of available blocks in a linked list, for example. In its simplest form, a single list is used to track free blocks. The allocator then sequentially considers each block and selects one according to a specified policy. Below, we will provide an overview of the most common policies used in free-list allocation.

\begin{description}
    \item[First-fit]
        The first block in the free-list that is large enough to fulfill the memory allocation request is selected. This method minimizes search time but does not consider the possibility of a more suitable block elsewhere in the list. This often leads to more fragmentation than is necessary, because blocks are split more often. New requests restart their search from the beginning of the list.
    \item[Next-fit]
        The search for a suitable block begins in the free-list, following a similar process to that described for first-fit. However, in subsequent searches, it resumes from where the previous search ended, improving efficiency when locating a new block. This strategy is based on the observation that smaller blocks tend to accumulate at the beginning of the free-list~\cite{gchandbook}, optimizing the search process by starting further into the list in each iteration.
    \item[Best-fit]
        The entire free-list is searched until the smallest available block that can fulfill a request is located. This method minimizes fragmentation by selecting the block that best matches the size of the requested memory, but it comes with the trade-off of increased search time.
\end{description}

The three policies described above have in common that when a block that is larger than requested is found, it is split up. Blocks are generally desired to be split as infrequently as possible to have larger blocks available for larger requests. If blocks are split too often, many small blocks might accumulate, which might increase external fragmentation to a level where new requests cannot be fulfilled. Additionally, splitting blocks less frequently will also mean less merging, or coalescing, of blocks to larger sizes, which could improve performance.

\subsubsection{Segregated-Fits Allocation}

Instead of using a single free-list where blocks of many sizes are stored, multiple free-lists that store blocks of similar sizes or size ranges can instead be used, called segregated-fits. The goal of using multiple free-lists is to narrow down the search space to fewer blocks, minimizing the time to find blocks large enough to satisfy a request. It is crucial to note that blocks are logically segregated into their respective free-lists based on size but are not required to be physically adjacent in memory within the same free-list. 

Segregated-fits is often employed in real-time systems where predictable and efficient memory allocation is crucial~\cite{gchandbook, TLSF}. The reduced search space and minimized search time to find suitable blocks increase the probability of timing constraints being met.

%%% Local Variables:
%%% mode: latex
%%% TeX-master: "main"
%%% End:


\newpage
\subsection{Garbage Collection}
\label{sec:gc}

This section draws heavily from \textit{The Garbage Collection Handbook: The Art of Automatic Memory Management}~\cite{gchandbook}, by R. Jones et al., which provides a comprehensive overview of garbage collection methods, both historical and current.

As mentioned in Section~\ref{sec:memory_management}, garbage collection is a form of automatic memory management where the system identifies and cleans up unused objects, which are considered garbage. This removes the requirement of managing memory manually, which reduces the possibility of user-related memory issues occurring. However, this also removes some control from the user and could lead to an increased memory footprint.

From the perspective of the GC, the user program can be perceived solely as a memory mutator. This is because the GC is only concerned with the memory management of the program, not its functionality. Therefore, the threads of the user program are commonly referred to as mutator threads, or just mutators.

During GC execution, it may be necessary to temporarily halt the mutator threads to give the GC exclusive memory access. This is known as a stop-the-world approach. From the mutator's perspective, the GC appears to be atomic. This permits a simple implementation and avoids synchronization between GC and mutators while the GC is running, but also introduces pauses in the application which may be undesirable if latency is a quality metric. The GC could also operate concurrently and in parallel with the mutator threads. This leads to increased complexity as the memory can mutate during GC execution. Most GCs fall somewhere in between these ends, executing concurrently with the mutator threads while occasionally requiring pauses in their execution.

There exist numerous garbage collection strategies. One is the mark-sweep algorithm, which begins by identifying a set of accessible ``root'' objects. These objects are then marked as live and scanned to identify other objects. This process continues recursively until all live objects have been identified. Afterwards, the heap is swept, freeing all objects that were not previously marked.

The mark-sweep algorithm does not relocate objects, which can result in higher levels of external fragmentation. The mark-compact algorithm addresses this issue. Like mark-sweep, it traverses the heap and marks all live objects. However, instead of sweeping, live objects are relocated and compacted. This approach reduces fragmentation and can be managed using a simple bump-pointer allocator. It may be necessary to repeat this process multiple times during a single GC cycle depending on the positioning of live objects, as they could potentially collide with the moving operation.

A copying garbage collector solves the issue of objects colliding when being moved. The copy algorithm divides the heap into many subheaps. A subheap is first chosen and marked as non-empty. Allocations are then made sequentially onto this subheap using bump-pointer allocation. When garbage collection is initiated, live objects are copied from their current subheap to another subheap. Once all live objects have been copied, the original subheap is marked as empty and its bump-pointer reset. Compared to mark-compact, heap size is significantly reduced when using few subheaps. Using many subheaps will make each subheap small, which could result in more garbage collection cycles.

%%% Local Variables:
%%% mode: latex
%%% TeX-master: "main"
%%% End:


\subsection{OpenJDK}
\label{sec:openjdk}

OpenJDK~\cite{openjdk} is a set of tools for creating and running Java programs, maintained by Oracle. HotSpot~\cite{hotspot} is one of these tools and the reference implementation of the Java Virtual Machine (JVM)~\cite{JVM}. Specifically, a virtual machine (VM) emulates a distinct computer to run various programs, which could be anything from full operating systems -- to more specialized machines. The JVM is designed specifically to execute Java programs. It translates programs to instructions for the underlying machine, creating an abstraction of the hardware of the physical machine and allows Java programs to be run on any platform that the JVM runs on.

HotSpot is comprised of several components, such as an interpreter, a Just-In-Time (JIT) compiler, and a garbage collector (GC). In combination, these components provide the means for running different types of Java programs on the platforms supported by HotSpot.

HotSpot provides several garbage collectors, each with different characteristics and performance profiles. Different garbage collectors are optimized for different use cases, such as throughput and latency, and can be tuned to varying degrees. One of the garbage collectors in HotSpot is the Z Garbage Collector (ZGC)~\cite{zgc}. It was introduced as an experimental feature in OpenJDK 11, and declared production ready in OpenJDK 15. ZGC is a modern, generational, region-based, concurrent garbage collector that aims to keep pause times constant at any given heap size.

%%% Local Variables:
%%% mode: latex
%%% TeX-master: "main"
%%% End:



\subsection{The Z Garbage Collector}
\label{sec:zgc}

The following sections will cover the basics of how ZGC allocates data and how it can dynamically collect garbage. The information is based partly on an overview by Yang and Wrigstad~\cite{zgc:deep_dive} and partly on the source code of ZGC itself, as available in OpenJDK version 22.32~\cite{jdk:tag2232}.

%%% Local Variables:
%%% mode: latex
%%% TeX-master: "main"
%%% End:


\subsubsection{Pages}
\label{sec:zpage}

ZGC is a region-based garbage collector which allocates objects inside differently-sized regions of memory that are referred to as pages. When new objects are created, they are allocated inside pages with respect to their size. Pages are classified as one of three different types: \textit{Small}, \textit{Medium} or \textit{Large}, which support allocations of different size ranges, as specified in Table~\ref{table:zpage_sizes}. All allocations are aligned to 8 bytes and the smallest supported allocation size is, at the time of writing, 16 bytes. As a result of all allocations having the same alignment, their offset from the start of the page will also be a multiple of 8 bytes.

\begin{table}[H]
    \centering
    \begin{tabular}{lllll}
        Page Type & Page Size   & \multicolumn{3}{l}{Object Size}      \\ \hline
        Small     & 2 MB        & \multicolumn{3}{l}{{[}16B, 256KB{]}} \\
        Medium    & 32 MB       & \multicolumn{3}{l}{(256KB, 4MB)}     \\
        Large     & $\geq$ 4 MB & \multicolumn{3}{l}{$\geq$ 4MB}       \\
    \end{tabular}
    \caption{Page sizes in ZGC (Taken from Yang and Wrigstad~\cite{zgc:zpage_size_table}).}
    \label{table:zpage_sizes}
\end{table}

Pages in ZGC include metadata that allows them to perform allocations efficiently, as well as ensure safe execution while running concurrently executing threads. An overview of this metadata is shown in Figure~\ref{fig:zpages}, where the most important attributes are the: \textit{Bump Pointer}, \textit{Live Map}, \textit{Sequence Number}, and \textit{Age}, which are explained in detail below.

\begin{description}
    \item[Bump Pointer]
        The bump pointer is a pointer within the page's memory range that stores the location of where new allocations are made inside the page. Its functionality and use-case are explained in detail in Section~\ref{sec:bump_pointer}. All allocations inside pages in ZGC are currently done using bump pointers. In Figure~\ref{fig:zpages}, the bump pointer is displayed below four allocated objects, indicating that the next position to allocate objects at would be below the fourth object.
        
    \item[Live Map]
        The live map is used to keep track of which objects are currently live and have been marked as reachable by the program. Objects not marked in the live map are considered dead. The live map is shown on the right-hand side of Figure~\ref{fig:zpages}. It is constructed during the marking phase of the GC cycle as live objects are encountered during memory traversal and marked in the live map. The live map is represented by a bitmap, where each bit is mapped to an 8-byte chunk of the page. If a bit is set, there is a live object at the corresponding address on the page. The live map is used during the compacting phase of the GC in order to know how much memory is alive inside the page.
        \newpage
    \item[Sequence Number]
        Every page has a sequence number denoting during which GC cycle it was created. If the GC is on its $N$th cycle, pages created during that GC cycle will have a sequence number $N$. Pages with sequence number $N$ are called \textit{allocating} pages, and are the only pages which allocate new objects for mutator threads. Pages with a sequence number below the current GC cycle, $0$--$N-1$ are called \textit{relocatable} pages since there will be no additional allocations done inside them during the GC cycle, and garbage can be reclaimed with the use of relocation. For example, if the current Java program has performed 5 GC cycles, any allocations after that will be exclusively done on pages with sequence number 5. If the program decides to run a 6th cycle, only garbage from pages with sequence numbers 0 through 5 will be collected, and new allocations will be done on pages with sequence number 6.
        
    \item[Age]
        ZGC is a generational garbage collector, meaning it treats objects of varying ages differently to improve performance. This is based on the observation that most objects survive for only a short period and those that survive for longer tend to live for a long time. ZGC applies this approach on a page-by-page basis, where all allocations on a page share the same age, which is conveniently stored on each page instead of for every object. Objects are grouped into three different age categories: \textit{Eden}, \textit{Survivor} and \textit{Old}. The \textit{Eden} age is for first-time allocation, \textit{Survivor} signifies objects that have survived one or more GC cycles and \textit{Old} those that have survived a threshold of cycles in the \textit{Survivor} age. Classifying pages this way allows the GC to treat pages of a certain age differently, which is especially useful for \textit{Eden} pages, which tend to accumulate garbage quickly.
\end{description}

\begin{figure}[H]
    \centering
    \includesvg[scale=0.8]{figures/zpage_withage.svg}
    \caption{Illustration of a page in ZGC showing what kind of metadata is kept track of to facilitate allocation and object bookkeeping. The page contains three live objects marked in green and with the corresponding bit set in the live map to the right. Red objects are previously allocated objects which have since become unreachable garbage memory.}
    \label{fig:zpages}
\end{figure}

\subsubsection{The Garbage Collection Cycle}

Figure~\ref{fig:zgc_timeline} shows a timeline of a garbage collection cycle in ZGC, made up of an example heap and pages. The timeline shows the different phases of a cycle and how the garbage collector prepares for relocating memory to free unused memory. The three different steps in the timeline show:

\vspace*{-0.4cm}

\begin{enumerate}[label=\alph*)]
    \item The initial state of the heap before the GC cycle starts. The left page has about 30\% free memory and the right page has about 50\%. The current GC cycle is 1, which tells us that pages with sequence number 1 are allocating objects.
    \item The GC cycle has started, and the old pages have been locked. New allocations are now done on new pages instead of the old ones. The old pages are now guaranteed to not allocate any new objects, which makes it possible for the marking phase to begin to traverse the live objects.
    \item The GC has finished the marking phase and now has liveness data that indicate where live objects are allocated inside the relocatable pages.
\end{enumerate}

\vspace*{-0.4cm}

\begin{figure}[H]
    \centering
    \begin{subfigure}[t]{.214\textwidth}
        \centering
        \includesvg[width=1\textwidth]{figures/zrel1.svg}
        \caption{Initial state of the heap before the first GC cycle. The gray color indicates the page is allocating objects.}
        \label{fig:zrel1}
    \end{subfigure}
    \hfill\vline\hfill
    \begin{subfigure}[t]{.32\textwidth}
        \centering
        \includesvg[width=1\textwidth]{figures/zrel2.svg}
        \caption{The state of the heap after marking has started. The blue color indicates that the page is no longer being allocated on.}
        \label{fig:zrel2}
    \end{subfigure}
    \hfill\vline\hfill
    \begin{subfigure}[t]{.32\textwidth}
        \centering
        \includesvg[width=1\textwidth]{figures/zrel3.svg}
        \caption{The state of the heap after all live objects are marked. The red and green colors indicate that the marking phase has finished, and the pages have liveness data that explain where live objects(green) are located}.
        \label{fig:zrel3}
    \end{subfigure}
    \caption{Timeline of a garbage collection cycle, showing the state of the heap in the different phases.}
    \label{fig:zgc_timeline}
\end{figure}

\vspace*{-0.49cm}

ZGC frees memory marked as garbage by copying live objects from one page to another and re-purposing the source page for new allocations. This leads to the fragmented layout of allocated memory to be compacted which makes it possible to fit objects from multiple pages into a single one. The process of copying live objects is referred to as relocation. When performing relocation, two different scenarios can occur. Either the heap has enough free memory to allocate a new page of contiguous free memory to relocate objects to, or the heap is full, requiring the objects to be compacted into the same page that they are already located in, called in-place compaction. The first scenario is illustrated in Figure~\ref{fig:zrel_new}. This requires the heap to have enough free space available to create a new page during the time of relocation. The GC can reclaim garbage memory by relocating live objects from sparsely populated pages into a more dense configuration inside a new page. The fragmentation and also total heap usage is therefore reduced.

\begin{figure}[H]
    \centering
    \begin{subfigure}[t]{.2\textwidth}
        \centering
        \includesvg[width=1\textwidth,height=1.2885\textwidth]{figures/zrel_new1.svg}
        \caption{A page has been selected for relocation due to high fragmentation.}
        \label{fig:zrel_new1}
    \end{subfigure}
    \hfill\vline\hfill
    \begin{subfigure}[t]{.2\textwidth}
        \centering
        \includesvg[width=1\textwidth]{figures/zrel_new2.svg}
        \caption{A new page is created with a new sequence number that is used as a relocation target.}
        \label{fig:zrel_new2}
    \end{subfigure}
    \hfill\vline\hfill
    \begin{subfigure}[t]{.2\textwidth}
        \centering
        \includesvg[width=1\textwidth]{figures/zrel_new3.svg}
        \caption{The objects from the first page are relocated to the new page.}
        \label{fig:zrel_new3}
    \end{subfigure}
    \hfill\vline\hfill
    \begin{subfigure}[t]{.2\textwidth}
        \centering
        \includesvg[width=1\textwidth]{figures/zrel_new4}
        \caption{All objects have been copied to the new page and the old page is re-purposed.}
        \label{fig:zrel_new3}
    \end{subfigure}
    \caption{An example of how a successful relocation is done when the heap has enough space to allocate a new page.}
    \label{fig:zrel_new}
\end{figure}

Typically, many sparsely populated pages can be compacted onto a single new page, but this is conditional on there being enough memory to allocate the new page. Since we are starting GC as a reaction to high memory pressure, this may not always be the case. If a new page cannot be allocated, the first page is "created" by compacting its objects in-place, and then continuing by relocating the live objects of other pages onto that page. An in-place compaction is an expensive operation that re-arranges the objects of a page without a separate page as a destination. It requires the thread performing the re-arrangement to write to the memory of the page while other threads may read its contents. To do this safely, any threads trying to read from the page during this process must be paused, which removes concurrent execution of the page. The process of in-place compaction of a page is illustrated in Figure~\ref{fig:zrel_in}.

\begin{figure}[H]
    \centering
    \begin{subfigure}[t]{.2\textwidth}
        \centering
        \includesvg[width=1\textwidth,height=1.2885\textwidth]{figures/zrel_in1.svg}
        \caption{The first live object on the page is moved to the start of the page.}
        \label{fig:zrel_in1}
    \end{subfigure}
    \hfill\vline\hfill
    \begin{subfigure}[t]{.2\textwidth}
        \centering
        \includesvg[width=1\textwidth,height=1.2885\textwidth]{figures/zrel_in2.svg}
        \caption{The second live object is moved as close to the start as possible.}
        \label{fig:zrel_in1}
    \end{subfigure}
    \hfill\vline\hfill
    \begin{subfigure}[t]{.2\textwidth}
        \centering
        \includesvg[width=1\textwidth,height=1.2885\textwidth]{figures/zrel_in3.svg}
        \caption{All objects have been moved, and the space after is made available.}
        \label{fig:zrel_in1}
    \end{subfigure}
    \hfill\vline\hfill
    \begin{subfigure}[t]{.2\textwidth}
        \centering
        \includesvg[width=1\textwidth,height=1.2885\textwidth]{figures/zrel_in4.svg}
        \caption{The bump pointer is moved to the beginning of available space.}
        \label{fig:zrel_in1}
    \end{subfigure}
    \caption{An example of how in-place compaction is done to remove fragmentation.}
    \label{fig:zrel_in}
\end{figure}

%%% Local Variables:
%%% mode: latex
%%% TeX-master: "main"
%%% End:


\section{Related work}
Due to the large number of existing allocator designs, one might conclude that most problems of dynamic memory allocation have already been solved. For most cases, a general-purpose allocator designed to be used in any system or environment performs well on average, in terms of response time and fragmentation. However, there may exist areas of improvement, depending on how much the use case can be narrowed down. For example, a garbage collector would have access to more information about the objects being allocated than most other users of an allocator.

\subsection{Memory Allocators}

One of the first memory allocation techniques is the very simple buddy allocator, conceived by Knowlton \cite{buddy}. The buddy allocator partitions memory into blocks, each sized as a power of two. Initially, the entire memory is set as a single block. In the first allocation, this block is divided recursively into halves until the block size fits the allocation size closest. These new differently sized blocks can then be utilized for subsequent allocations. The two resulting blocks from the splitting of a block are called ``buddies''. When a block is deallocated and its buddy is also unallocated, the two blocks are combined into a larger block. This merging process continues recursively as long as the buddy is also unallocated.

The simple design of the buddy allocator provides ample room for enhancements. Peterson and Norman presented a more generalized version of the buddy allocator~\cite{genbuddy}, allowing block sizes that do not necessarily have to be powers of two and allowing for buddies of unequal sizes. Recent advances have also included making the allocator lock-free~\cite{nbbs} and enabling the same block to exist as multiple sizes simultaneously to improve the efficiency of smaller allocations~\cite{park2014ibuddy}.

The buddy allocator has seen wide use in real environments and applications. A notable example is its use within the Linux kernel for allocating physical memory pages~\cite{linuxbuddy}. In addition, the buddy allocator can be combined with various other allocation techniques to improve efficiency. For example, \texttt{jemalloc}~\cite{jemalloc}, created for the FreeBSD operating system, is a general-purpose allocator that incorporates multiple strategies, including the buddy allocator, to achieve overall efficiency.

\subsection{Free-Lists in Garbage Collection}

Using free-lists in garbage collection systems is not a new concept. For example, the basic mark-sweep algorithm, as outlined in Section~\ref{sec:gc}, necessitates the use of a free-list. The static position of live objects can lead to significant fragmentation. Consequently, simpler allocation methods, such as bump-pointer allocation, become impractical. Instead, employing a strategy based on free-lists for allocation proves to be a more effective approach to manage and allocate memory surrounding live objects.

Among the free-list strategies outlined in Section~\ref{sec:freelist_allocation}, the segregated-fit method is particularly noted for its effectiveness, as emphasized by R. Jones et al.~\cite{gchandbook}. Programs that anticipate garbage collection generally allocate more than those that manually handle memory. Therefore, the significance of segregated-fit free-lists becomes particularly crucial as program allocations increase.

The Concurrent Mark Sweep (CMS) garbage collector~\cite{cms}, introduced in JDK 1.4, stands as a concrete example of free-list being used within real-world garbage collectors. CMS implements a concurrent variant of the mark-sweep algorithm, thereby utilizing free-lists for managing memory. Although it was deprecated in JDK 9, CMS demonstrates the practicality of using free-lists in garbage collectors. The presence of free-lists, particularly in the context of Java program execution, highlights their versatility and adaptability across different garbage collection algorithms and implementations.

It is evident that there exists a space for free-list-based allocators to exist within garbage collection systems. Furthermore, we know that the use of multiple allocation strategies in tandem can enhance allocation efficiency. Therefore, it is reasonable to suggest implementing a free-list-based allocator, such as the buddy allocator, into a garbage collection system, such as ZGC, as an alternative strategy to boost performance in certain situations.

%%% Local Variables:
%%% mode: latex
%%% TeX-master: "main"
%%% End:


\newpage
\section{Binary Buddy Allocator}
\label{sec:buddy}
The concept of a buddy memory allocator was first introduced by Knowlton~\cite{buddy}. The core idea of the buddy allocator involves dividing memory into blocks and pairing them as ``buddies''. These buddy pairs are split when allocating memory and merged when deallocating memory to allow allocations of different sizes. The simple design of the buddy allocator makes it attractive for various applications. For example, the Linux kernel implements a modified binary buddy allocator to manage physical memory~\cite{linuxbuddy}, and \texttt{jemalloc} adopts a buddy allocator for one of its memory allocation methods~\cite{jemalloc}.

\subsection{Allocation and Deallocation}
The binary buddy allocator is the original and most basic type of buddy allocator. It partitions memory into blocks with sizes that are powers of 2. A free list of available blocks of different sizes is maintained to keep track of the free blocks. Initially, all memory is one large block. Figure~\ref{fig:buddystart} shows the initial state of a binary buddy allocator with a total memory of 128 bytes. It can be seen that the entire memory is contained within a single block, which is also included in the free list.

\begin{figure}[h]
    \centering
    \includesvg[width=0.65\textwidth]{figures/bbuddy_initial.svg}
    \caption{The initial state of the free-list and memory in a binary buddy allocator.}
    \label{fig:buddystart}
\end{figure}

When requesting memory, a block of the smallest size that can accommodate the requested memory will be returned. If there are no blocks of that size available, larger blocks will be recursively split into two until a block of the correct size is available. The split blocks are then added to the free list for future allocations. An algorithmic explanation of this process can be found in Algorithm~\ref{alg:bbuddy_alloc}.

\begin{algorithm}[h]
    \caption{Binary buddy allocation algorithm}
    \label{alg:bbuddy_alloc}
    \begin{algorithmic}[1]
        \Statex \textbf{Procedure} Allocate(size)
        \State $level \gets \text{smallest power of 2} \geq size$

        \Statex \textbf{Find a suitable block}
        \State $block \gets \text{no\_block}$
        \For{$i \gets level$ to \text{max\_level}}
        \If{\text{free list for level $i$ is not empty}}
        \State $block \gets \text{remove first block from free list of level } i$
        \State \textbf{break}
        \EndIf
        \EndFor

        \If{$block = \text{no\_block}$}
        \State \Return \text{allocation\_failed}
        \EndIf

        \Statex \textbf{Split the block if necessary}
        \While{$\text{level of } block > level$}
        \State $block, buddy \gets \text{split } block \text{ into two buddy blocks}$
        \State \text{add } $buddy$ \text{ to its corresponding free list}
        \EndWhile

        \State \Return $block$
    \end{algorithmic}
\end{algorithm}

As an example, consider Figure~\ref{fig:buddysplit} which shows the resulting state of the previous figure after requesting 16 bytes of memory. The initial 128-byte block has been split multiple times to fulfill the request, with the leftmost block being returned and the addition of various block sizes to the free list.

\begin{figure}[h]
    \centering
    \includesvg[width=0.65\textwidth]{figures/bbuddy_allocated.svg}
    \caption{The state of the free-list and memory in a binary buddy allocator after one 16-byte allocation.}
    \label{fig:buddysplit}
\end{figure}

\newpage
Deallocation is the reverse process of allocation. When a block is deallocated, it is merged with its buddy if the buddy is also deallocated. Merging creates a new, larger block. Afterward, the original block and its buddy are removed from the free list, and the new, larger block is inserted. The larger block checks the status of its buddy and the merging process continues recursively until merging is no longer possible. An algorithmic explanation of this process can be found in Algorithm~\ref{alg:bbuddy_dealloc}. To demonstrate this process, consider the situation where the 16-byte block in Figure~\ref{fig:buddysplit} is deallocated. In this scenario, the block would go through a recursive merging process with its buddies until it reaches the initial state shown in Figure~\ref{fig:buddystart}.

\begin{algorithm}[h]
    \caption{Binary buddy deallocation algorithm}
    \label{alg:bbuddy_dealloc}
    \begin{algorithmic}[1]
        \Statex \textbf{Procedure} Deallocate(block)
        \State $level \gets \text{level of } block$

        \While{true}
        \State $buddy \gets \text{find buddy of } block$
        \If{$\text{buddy is not free}$}
        \State $\text{add } block \text{ to free list of level } level$
        \State \textbf{break}
        \Else
        \State $\text{remove } buddy \text{ from free list of level } level$
        \State $block \gets \text{merge } block \text{ and } buddy$
        \State $level \gets level + 1$
        \EndIf
        \EndWhile
    \end{algorithmic}
\end{algorithm}
\FloatBarrier

\subsection{Determining Buddy State}
When deallocating a block, the allocator must determine the state of its corresponding buddy. This is done using a bitmap, each entry indicating if a block has been allocated or not. The allocator assigns a unique number to each block that can potentially be allocated. Figure~\ref{fig:buddyorder} provides an illustration of all possible blocks and their respective numbering. The bitmap requires a total of $2^n - 1$ entries, where $n$ is the number of levels of potential block sizes.

\begin{figure}[h]
    \centering
    \includesvg[width=0.525\textwidth]{figures/bbuddy_order.svg}
    \caption{Unique numbering of each possible block in a binary buddy allocator.}
    \label{fig:buddyorder}
\end{figure}
% \vspace{-0.7cm}
The bitmap is correlated with the available blocks in the free-list and is modified both when splitting and merging blocks. For example, Figure~\ref{fig:buddybmapallocated} shows the state of the bitmap of a binary buddy allocator after one 16-byte allocation. We can see that the largest block has been split down to the smallest size, and that the buddies of the used blocks are marked in the bitmap as available.
% \vspace{-0.1cm}

% The bitmap is correlated with the available blocks in the free-list and is modified both when splitting and merging blocks. Figure~\ref{fig:buddybmap} shows the initial state of the bitmap when a binary buddy allocator is created. We can see that only the single largest block is marked as free, with all other blocks not being available. Figure~\ref{fig:buddybmapallocated} shows the state of the bitmap after the same previous 16-byte allocation. We can now see that the buddies of the split blocks are marked as available and that the large block has been removed.

% \begin{figure}[H]
%     \centering
%     \includesvg[width=0.9\textwidth]{figures/bbuddy_bmap_initial.svg}
%     \caption{Initial state of the bitmap and free-list of a binary buddy allocator.}
%     \label{fig:buddybmap}
% \end{figure}

\begin{figure}[h]
    \centering
    \includesvg[width=0.65\textwidth]{figures/bbuddy_bmap_allocated.svg}
    \caption{The state of the bitmap and free-list of a binary buddy allocator after one 16-byte allocation.}
    \label{fig:buddybmapallocated}
\end{figure}

%%% Local Variables:
%%% mode: latex
%%% TeX-master: "main"
%%% End:


\newpage
\section{Improved Buddy Allocators}
\label{sec:buddyadditions}
As of the time of writing, the original binary buddy allocator is nearly 60 years old. Over the years, numerous improvements and advances have been made, including refining the initial design~\cite{genbuddy} and adapting it for new applications and scenarios~\cite{nbbs}~\cite{park2014ibuddy}.

\subsection{Binary Tree Buddy Allocator}
The original binary buddy allocator keeps track of available blocks by organizing them into different free-lists. Consequently, every time an allocation or deallocation occurs, entries must be added to and removed from these lists. If a block needs to be split or can be merged, even more operations are needed. To avoid these costly free-list manipulations, an implicit free-list can be implemented using a binary tree structure. This tree includes all the information needed to recreate the original free-list.

An example of such an implementation is that of Restioson~\cite{btbuddy}. However, it should be noted that while Restioson offers a detailed and functional example of the technique, it may not necessarily reflect the original inception of the idea. This source is cited as the best available reference for this particular implementation, given the limitations in tracing the precise origins of the method.

The tree stores the maximum allocation size for each block or its children. Since blocks are evenly divided into two, only the block's ``level'' needs to be stored to determine its size. The initial state of a binary tree buddy allocator is illustrated in Figure~\ref{fig:btbuddyinitial}. Each level of the tree represents a specific size, and the root level corresponds to the entire memory space.

\begin{figure}[h]
  \centering
  \includesvg[width=0.73\textwidth]{figures/btbuddyinitial.svg}
  \caption{The initial state of the binary tree and its free-list equivalent in a binary tree buddy allocator.}
  \label{fig:btbuddyinitial}
\end{figure}

\subsubsection{Allocation in a Binary Tree Buddy Allocator}
When allocating with a binary tree, the level required for the allocation size is first determined. Then, a binary search is started from the root node of the tree. The node is checked to see if the available level is high enough; if it is, it checks the left child. If the left child does not have a large enough block available, the right child is chosen instead. This search process continues recursively on the child node until the correct level within the tree is reached. Upon finding a suitable node, its level is set to 0, and the parent nodes are updated to reflect the new state of their children. An algorithmic explanation of this process can be found in Algorithm~\ref{alg:btbuddy_alloc}.

\begin{algorithm}
  \caption{Binary tree allocation algorithm}
  \label{alg:btbuddy_alloc}
  \begin{algorithmic}[1]
    \Statex \textbf{Procedure} Allocate($size$)
    \State $level \gets \text{smallest power of 2} \geq size$

    \State $node \gets \text{root of tree}$
    \Statex \textbf{Find an available node}
    \While{$\text{level of } node > level$}
    \State $left, right \gets \text{get children of } node$
    \If{$\text{level avaliable in } left \geq level$}
    \State $node \gets left$
    \Else
    \State $node \gets right$
    \EndIf
    \EndWhile

    \If{$\text{level avaliable in } node < level$}
    \State \Return \text{allocation\_failed}
    \EndIf

    \State $\text{mark } node \text{ as unavaliable}$
    \Statex \textbf{Update the tree}
    \ForAll{$parent \text{ nodes starting from } node$}
    \State $\text{level available in } parent \gets \text{max(}left,right\text{)}$
    \EndFor
    \State \Return $node$
  \end{algorithmic}
\end{algorithm}

To illustrate this, Figure~\ref{fig:btbudydallocated} shows the state of a binary tree buddy allocator following the allocation of a 32-byte and a 16-byte block. It can be seen that the initial allocation is placed in the leftmost node at the second-lowest level, while the subsequent allocation is placed to the right of it at the lowest level. Consequently, the entire left branch of the binary tree now only contains one 16-byte block, as indicated by one being stored in the respective nodes.

\begin{figure}[h]
  \centering
  \includesvg[width=0.73\textwidth]{figures/btbuddyallocated.svg}
  \caption{The state of the binary tree and its free-list equivalent in a binary tree buddy allocator after one 32-byte and one 16-byte allocation.}
  \label{fig:btbudydallocated}
\end{figure}

\subsubsection{Deallocation in a Binary Tree Buddy Allocator}
Deallocation in a binary tree buddy allocator is also a simple process. Initially, the node that has been deallocated is set to its corresponding level, followed by updating its parent nodes to reflect this change. When the buddy of a block is also free, signified by it having the same level as the recently deallocated node, they are merged. The merging operation in a binary tree involves increasing the level of the parent by one, indicating that the entire memory space it represents is now available. An algorithmic explanation of this process can be found in Algorithm~\ref{alg:btbuddy_dealloc}.

\begin{algorithm}
  \caption{Binary tree deallocation algorithm}
  \label{alg:btbuddy_dealloc}
  \begin{algorithmic}[1]
    \Statex \textbf{Procedure} Deallocate($block$)
    \State $level \gets \text{level of } block$
    \Statex \textbf{Set the level of the node}
    \State $node \gets level$

    \Statex \textbf{Update the tree}
    \ForAll{$parent \text{ nodes starting from } node$}
    \If{$left = right \text{ and } left \text{ is free completely}$}
    \State $\text{level available in } parent \gets left + 1$
    \Else
    \State $\text{level available in } parent \gets \text{max(}left,right\text{)}$
    \EndIf
    \EndFor
  \end{algorithmic}
\end{algorithm}

\subsection{Inverse Buddy Allocator}
Allocating the smallest block size in a binary buddy allocator requires the most total block splits. To improve the speed of allocating these blocks, adjustments are necessary. Park et al.~\cite{park2014ibuddy} proposed an alternative method for organizing the metadata of a binary buddy allocator, which they call an inverse buddy allocator (iBuddy allocator). An iBuddy allocator can allocate individual blocks in constant time, although at the expense of slower allocations for larger blocks. Rather than dividing larger blocks into smaller ones to fulfill allocations, all blocks are initially split, and merging is only done during larger allocations.

\subsubsection{Allocation in an Inverse Buddy Allocator}
In an iBuddy allocator, blocks of different sizes that overlap the same memory are all present in the free-list and bitmap at the same time. Consider Figure~\ref{fig:ibuddyinitial}, which shows a valid initial state of an iBuddy allocator. It can be seen that all potential addresses for the smallest-size blocks are present in the free-list but at different levels. In the iBuddy allocator, the bitmap takes on additional meaning. When a block is marked as free, it implies that all smaller blocks at that memory location are also free and available for allocation.

\begin{figure}[h]
  \centering
  \includesvg[width=0.702\textwidth]{figures/ibuddy_bmap_initial.svg}
  \caption{The initial state of the free-list and bitmap in an inverse binary buddy allocator.}
  \label{fig:ibuddyinitial}
\end{figure}

Due to the new structure of the bitmap, it is possible to allocate single-sized blocks in place of larger blocks without affecting other blocks in the bitmap. When using an iBuddy allocator, the largest available block in the free-list is always utilized for allocation, irrespective of the allocation size. Consider Figure~\ref{fig:ibuddyallocated}, which shows an allocation of one block of the smallest size. The largest block from the preceding figure has been taken out of the free-list and bitmap, whereas the remainder of the free-list and bitmap remains unaltered. This allocation did not influence other blocks, and all other smallest-size blocks are still accessible at different levels. No explicit block splitting is needed, and by storing the highest level of the free-list with available blocks, the allocation of smallest-sized blocks can be done in constant time.

\begin{figure}[h]
  \centering
  \includesvg[width=0.702\textwidth]{figures/ibuddy_bmap_allocated.svg}
  \caption{The state of the free-list and bitmap in an inverse binary buddy allocator after one
    16-byte allocation.}
  \label{fig:ibuddyallocated}
\end{figure}

Fast allocation speed for small blocks comes at the cost of slow allocation speed for larger blocks. In contrast to a binary buddy allocator, blocks cannot simply be removed, as there are now overlapping smaller blocks present in the free-list and bitmap. These must be removed to avoid allocating the same memory location twice. An algorithmic explanation of this process can be found in Algorithm~\ref{alg:ibuddy_alloc}.

Consider Figure~\ref{fig:ibuddyallocated2}, which shows the same allocator as in the previous figure, now with an additional 64-byte allocation. It can once again be seen that the highest-level block has been removed, but now, all the blocks under it have also been cleared from both the free-list and the bitmap. For larger blocks, the allocation speed is proportional to the number of smallest-sized blocks needed to fill the requested size.

\begin{algorithm}[h]
  \caption{iBuddy allocation algorithm}
  \label{alg:ibuddy_alloc}
  \begin{algorithmic}[1]
    \Statex \textbf{Procedure} Allocate($size$)
    \State $level \gets \text{smallest power of 2} \geq size$
    \State $block \gets \text{remove the first block from the highest-level non-empty free list}$

    \If{$level = 0$}
    \State \Return $block$
    \ElsIf{$\text{level of } block < level$}
    \State \Return \text{allocation\_failed}
    \EndIf

    \Statex \textbf{Remove all lower blocks}
    \For{$i \gets level$ to \text{0}}
    \ForAll{$\text{possible blocks in level } i$}
    \State $\text{remove } block \text{ from free list at level } i$
    \EndFor
    \EndFor

    \State \Return $block$
  \end{algorithmic}
\end{algorithm}

\begin{figure}[h]
  \centering
  \includesvg[width=0.635\textwidth]{figures/ibuddy_bmap_allocated2.svg}
  \caption{The state of the free-list and bitmap in an inverse binary buddy allocator after one
    16-byte allocation and one 64-byte allocation.}
  \label{fig:ibuddyallocated2}
\end{figure}


\subsubsection{Deallocation in an Inverse Buddy Allocator}
In an inverse buddy allocator, no blocks are explicitly merged during deallocation; instead, the level at which they are placed back into the free-list indicates the size of the available block. During deallocation, the status of the block's buddy is checked; if it is free, instead of merging, the deallocated block rises one level. This process repeats until a buddy is no longer free, after which the block is inserted at that layer. An algorithmic explanation of this process can be found in Algorithm~\ref{alg:ibuddy_dealloc}.

As an example of the deallocation process, consider Figure~\ref{fig:ibuddydeallocated}, which shows the previous figure, now after deallocating the first 16-byte allocation. The buddy is first checked at the bottom level, then again at the level above that, to finally end up at the third level, where it is inserted and marked in the bitmap as free. The complexity of deallocating single blocks is the same as that of the binary buddy allocator, but less work needs to be done, as no explicit merging is done, only checks for the blocks' buddies.

\begin{figure}[h]
  \centering
  \includesvg[width=0.635\textwidth]{figures/ibuddy_bmap_deallocated.svg}
  \caption{The state of the free-list and bitmap in an inverse binary buddy allocator after one
    16-byte allocation, one 64-byte allocation, and one 16-byte deallocation.}
  \label{fig:ibuddydeallocated}
\end{figure}

Similar to allocation, the deallocation speedup for smaller blocks comes at the cost of larger blocks. When deallocating a large block, all the smaller blocks within that block also need to be inserted back into the free-list and bitmap. Thus, the operation of deallocating a large block is equivalent in cost to individually deallocating each of the smallest-size blocks that make up the larger block. This is more costly than a binary buddy, where larger blocks require the least work as they merge the fewest number of times.

\begin{algorithm}[H]
  \caption{iBuddy deallocation algorithm}
  \label{alg:ibuddy_dealloc}
  \begin{algorithmic}[1]
    \Statex \textbf{Procedure} Deallocate($block$)
    \ForAll{$\text{lowest-level blocks } current\_block \text{ that fit in } block$}
    \State $level \gets 0$
    \State $buddy \gets \text{find buddy of } current\_block \text{ at } level$
    \While{$buddy \text{ is free}$}
    \State $level \gets level + 1$
    \State $buddy \gets \text{find buddy of } current\_block \text{ at } level$
    \EndWhile
    \State $\text{add } current\_block \text{ to free list at level } level$
    \EndFor

  \end{algorithmic}
\end{algorithm}

\newpage
\section{Adapting the Buddy Allocator}
\label{sec:adaptations}
As seen in Section~\ref{sec:buddyadditions}, there is ample room to improve the original binary buddy allocator. The two designs discussed have different strengths, so both are implemented to compare and contrast with each other and the original design. The binary tree allocator is closer to the binary buddy allocator in how it allocates, whereas the ibuddy allocator is intrinsically different. Ideally, the allocation and deallocation processes should be most efficient for the most frequently requested allocation size, which is why both allocators are investigated.

Further enhancements can then be developed that are orthogonal to the specific allocator implementations. Such enhancements can be either general optimizations or optimizations specifically tailored to the context of operating within ZGC. Constricting the use case enables the allocators to assume access to additional information or to narrow down their functionality.

\subsection{Use Cases}
There exist numerous scenarios in which a garbage collector could benefit from an advanced memory allocator. There are likely many that will not be brought up in this thesis. Each scenario has its own distinct characteristics with unique requirements. Therefore, any improvements made should enhance the specific scenario without causing adverse effects in other situations. In case a modification does have negative repercussions on other scenarios, it must be possible to disable that change so that each scenario can utilize the most suitable configuration for its needs.

\newpage
Making the allocator configurable in this way improves its adaptability to a wide range of scenarios. Although this versatility is advantageous, it may lead to limitations in certain instances. Prioritizing configurability hinders the allocator from being fully optimized for one particular use case. Implementing significant modifications to enhance one specific scenario could have adverse effects on others. Such changes were not taken into account during the allocator's adaptation as a result of their perceived narrow focus.

\subsection{Allocator Metadata Implementations} \label{sec:adaptationsmetadata}
The binary tree allocator keeps the design of the original binary buddy allocator, but stores its metadata in a different format. This allows for faster operations, as no explicit free-list is kept that needs to be inserted and removed from.

The binary tree is stored as a flattened byte array. The start of the array aligns with the start of the first level, with each subsequent level stored contiguously following the preceding one. Although most levels have each byte representing the value of a single block, the lowest levels deviate from this pattern. Given that each node in the tree holds the maximum possible level that can be allocated within it or its children, the smallest blocks can only store 1 or 0. For the lowest level, memory usage can thus be optimized by compacting and storing 8 blocks within each byte in the array. As this level comprises half of the total blocks, this results in a memory saving of $\frac{1}{2}\times\frac{7}{8}=43.75\%$. Similarly, the data for the subsequent two levels can be compacted to 2 bits per block. This further reduces memory by $\frac{1}{4}\times\frac{6}{8}+\frac{1}{8}\times\frac{6}{8}=28.13\%$. Levels above these require at least 4 bits of information and constitute a smaller fraction of total blocks. The decision was made not to optimize the memory of these levels, as the performance impact of having to perform bit operations at each block would outweigh the negligible improvement in memory usage.

\subsection{Allocating on Already Used Memory} \label{sec:freerangeexpl}
There may be situations in which the use of an allocator may not be the most efficient option. For instance, one could opt for bump-pointer allocation initially and then switch to a more advanced allocator when significant external fragmentation arises. This approach allows for maximizing the speed advantage of bump-pointer allocations for as long as possible. In the context of ZGC, this means that the allocator is used selectively in situations where it is considered more effective than the usual bump-pointer operation.

\newpage
Normally, an allocator has full authority over its allocatable memory region, allowing it to store data and allocate memory locations as needed. However, in this particular scenario, the allocator must be able to allocate around occupied chunks within its memory area. This imposes certain limitations on the allocator, specifically preventing it from assuming that any random memory is available for its utilization. This affects the possibilities of where to store the allocator metadata, which is commonly stored within its memory region.

When working with previously allocated memory, the allocator is initially fully occupied, leaving no available locations for allocation. The allocator then requires the user to indicate either the specific locations of the occupied memory or the intervals between them. This process essentially involves the user defining free regions within the memory space that the allocator can utilize. In ZGC, the live analysis of a page contains information on live objects and can be used to find the specific memory locations that are fixed, enabling the deallocation of the intervals between them.

To implement this, the initial internal state of the full allocator is such that the entire memory is filled with smallest-sized blocks. This means that all blocks are split initially, which avoids accidentally merging blocks that overlap with occupied memory regions. When deallocating a specific range, the largest blocks that can fit into that range are added to the free-list, and each block and its split children are marked as no longer split. Figure~\ref{fig:deallocrange} shows an example of this, where only the last block is occupied. We can see that the blocks added to the free-list cover the entire free range and are of the maximum possible size.

\begin{figure}[h]
    \centering
    \includesvg[width=0.7\textwidth]{figures/deallocrange.svg}
    \caption{The blocks added to the free-list after deallocating the range up until the last block.}
    \label{fig:deallocrange}
\end{figure}

\subsection{Finding Block Buddies} \label{sec:findbuddiesexpl}
When deallocating, the allocator must determine the correct block to deallocate, given the memory address. However, a particular address can come from any block level. The simplest solution is to require the user to provide additional information about the allocated size, and then use that to calculate from which level to deallocate from. This is very fast and, if possible for the user, leads to the lowest overhead.

Another solution is to store the level of each allocation, either inside the block or outside it, in a separate data structure. This is not very memory efficient, but is simple and lookup is fast. When the data is stored within a block, the usable space becomes smaller and of inconvenient size. If the data is stored separately, each possible block location can instead be stored, resulting in a slightly larger but constant overhead.

A third way is to use a bitmap to track which blocks have been split. With this information, it is possible to deduce the size of a block at a particular address. This is very memory-efficient, but more operations are needed to find the correct blocks. Figure~\ref{fig:buddybmapsplit} shows how this bitmap looks after the same previous 16-byte block allocation. We can see that all blocks above that block have been marked as split. We can also see that the smallest blocks do not require an entry in the bitmap as they cannot be split. When a block is deallocated, the allocator will start at the top of the bitmap, then traverse down until the block is no longer marked as split.

\begin{figure}[h]
    \centering
    \includesvg[width=0.7\textwidth]{figures/bbuddy_bmap_split.svg}
    \caption{The state of the split blocks-bitmap and free-list of a binary buddy allocator after one 16-byte allocation.}
    \label{fig:buddybmapsplit}
\end{figure}

All the options mentioned are implemented and can be used in the allocators.  The most suitable option depends on the specific usage scenario, each having trade-offs between memory usage and processing speed. For the most general use case, the most efficient selection would be the bitmap approach, which is very memory efficient across many allocator configurations.

For use within ZGC, it is possible to omit the data structures responsible for tracking size and delegate this function to the garbage collector. The collector possesses sufficient data from its live analysis and the object headers to compute the size of each object and allocation. Given that the data is easily accessible, there is no need for the allocator to incur additional overhead, instead making optimal use of existing resources.

\subsection{Lazy Splitting and Merging of Blocks} \label{sec:lazyexpl}
Splitting and merging blocks is costly and should therefore be avoided if possible. The binary buddy allocator merges blocks whenever possible, which often results in the need to immediately split blocks on subsequent allocation. Lee and Barkley \cite{lazylayer} have designed a modified buddy allocator that dynamically delays the merging of blocks. As long as the allocation size distribution stays consistent, one can avoid the costs related to splitting and merging. To reduce this overhead in the modified allocators, a simpler version of Lee and Barkley's design is implemented. A second free-list (the lazy layer) is created on top of the buddy allocator (the buddy layer). This separate free-list is independent of the free-list of the buddy layer, and it does not perform any splitting or merging; it solely stores blocks of different sizes.

When deallocating using a lazy layer, the block is inserted into the free-list of the lazy layer instead of the buddy layer. However, if the size of the lazy layer reaches a certain threshold of blocks already in the lazy layer, the block is inserted into the buddy layer as normal. During allocation, the lazy layer is first checked for a block of the correct size. Only if this fails, the blocks from the buddy layer are split. If both of these steps fail, the lazy layer can be emptied to allow for the merging of smaller blocks to meet the required allocation size.

It is logical for different block sizes to have different thresholds, as the frequency of sizes differ. Since most allocations are small, it is also logical that that size should have a higher threshold. Storing large blocks in the lazy layer would also reserve these large blocks for only that size, which could lead to smaller allocations not being able to be fulfilled. When implementing this, the smallest block size has the highest threshold, with each level above that halving the threshold of the previous level. The default threshold is set to 1000. A high value is desired, as ZGC can often deallocate many objects to then allocate many objects of the same size. The lazy layer needs the capacity to support this, which is why the threshold is set relatively high.

% \subsection{Skewed Allocation Distribution} \label{sec:skewedexpl}
% In the binary buddy allocator, smaller blocks require more work due to the need for more splits and more potential merges. This contrasts with real-world scenarios, where smaller allocations are more frequent and larger ones are less common. Ideally, the allocation and deallocation processes should be most efficient for the most commonly requested allocation size.

\subsection{Allocator Regions} \label{sec:concurrencyexpl}

In the original binary buddy allocator, the size of the largest block is equal to the entire provided memory, and is then immediately split down into smaller blocks when allocating. A consequence of this is that when less memory is used, the largest blocks become very large. This is a problem in multithreaded programs, where many allocations can happen concurrently. Only one allocation can split the large block, leading to degraded performance with more concurrent allocations.

A way to avoid this is to set the maximum block size lower than the entire memory size and then have it consist of many regions of the same size, with each region containing the normal buddy block structure. Park et al.~\cite{park2014ibuddy} suggest doing this for the iBuddy allocator. For example, one could split the memory and have 2 regions, each having a maximum block size of $\frac{1}{2}$th the entire memory size, which would allow for two allocations concurrently. Diving the memory into regions would both limit the possible number of merges, increasing performance, and make the memory less fragmented if allocations to some regions are prioritized.

The maximum allocation size in a small ZGC page is 256 KB, which is a fraction of the page size of 2 MB. This allows for 8 allocations of the maximum size within a page. Therefore, the allocator is not required to handle blocks that exceed $\frac{1}{8}$th of the total memory size. Instead, the allocator can be divided into eight distinct regions, each covering $\frac{1}{8}$th of a page.

Each of these areas operate independently of all others, enabling them to operate concurrently without the need for extra logic. As long as the allocations are distributed evenly across the regions, they can run concurrently. This scalability extends up to the number of regions, with further allocations having to wait until a prior allocation within a region is completed.

There may be instances where certain regions fill more than others, leading to a decrease in overall potential throughput. When a thread initiates an allocation, it is assigned a specific region. If another thread is currently making an allocation in that region, it proceeds to check the subsequent regions. If all regions are already in use, the thread must wait for the allocation process in a region to finish. When entering a region, the necessary allocation logic is performed. In the case where an allocation is unsuccessful, for instance, due to insufficient space in the region, the thread exits that region and retries in the next one. A request for allocation fails when all regions are unable to accommodate the request.

% what is the problem
% how can it be solved
% why is it better

\section{Implementing Allocator Adaptations}
\label{sec:adaptationsimpl}
\subsection{Lazy Allocation Layer}
As discussed in Section~\ref{sec:lazyexpl}, splitting and merging blocks should be minimized, and a solution to this is to use a lazy layer. The implemented lazy layer contains a free-list of only the smallest few block sizes. These are the most commonly used sizes, so should be prioritized. In addition, storing large blocks in the lazy layer would reserve these large blocks for only that size, which could lead to smaller allocations not being able to be fulfilled. 

Additionally, each block size in the lazy layer has a different threshold before blocks are inserted into the buddy layer. Since most allocations are small, it is logical that that size should have a higher threshold. The thresholds are set according to the distribution of allocation sizes, with the smallest block size having a threshold of $1 000$, with the larger blocks a proportion of this value. ZGC can often deallocate many objects to then allocate many objects of the same size. The lazy layer needs the capacity to support this, which is why the threshold has been set very high. 

\subsection{Deallocating Ranges}
As explained in Section~\ref{sec:freerangeexpl}, the allocator may be completely occupied initially, with the ability to later clear ranges within its memory region to accommodate new allocations. To achieve this, the allocator is full initially and sections within it can be marked as available for the allocator to utilize.

The initial internal state of the full allocator is such that the entire memory is filled with the smallest-sized blocks. This means that all blocks are split initially, which avoids accidentally merging blocks that overlap with occupied memory regions. When deallocating a specific range, the largest blocks that can fit into that range are added in the free-list, and each block and its split children are marked as no longer split. Figure~\ref{fig:deallocrange} shows an example of this, where only the last block is occupied. We can see that the blocks added to the free-list cover the entire free range and are of the maximum possible size.

\begin{figure}[H]
    \centering
    \includesvg[width=0.9\textwidth]{figures/deallocrange.svg}
    \caption{The blocks added to the free-list after deallocating the range up until the last block.}
    \label{fig:deallocrange}
\end{figure}

\subsection{Determining Allocation Sizes}
As outlined in Section~\ref{sec:findbuddiesexpl}, there are different methods to track allocation sizes. The allocator currently implements all the mentioned alternatives, each with its own set of pros and cons. The most suitable option depends on the specific usage scenario, each having trade-offs between memory usage and processing speed. For the most general use case, the most efficient selection would be the bitmap approach, which is very memory efficient across many allocator configurations.

Storing block levels is efficiently implemented when the total number of levels is 16 or fewer. Having a lower number of levels reduces the required bits for storage, enabling each byte to accommodate two blocks rather than just one.

The implementation storing block levels is optimized if the total number of levels is 16 or fewer. Having a lower number of levels reduces the required bits for storage, enabling each byte to accommodate two blocks rather than just one.


For use within ZGC, it is possible to omit the data structures responsible for tracking size and delegate this function to the garbage collector. The collector possesses sufficient data from its live analysis and the object headers to compute the size of each object and allocation. Given that the data is easily accessible, there is no need for the allocator to incur additional overhead, instead making optimal use of existing resources.

\subsection{Alternative buddy designs}
Exploring ibuddy/binary tree buddy compared to original binary buddy


\subsection{Allocator Regions}
Split allocator into regions



\section{Theory}
\label{sec:theory}
\input{text/theory}

\section{Method}
\label{sec:method}
The methodology was divided into five distinct phases:

\begin{enumerate}
    \item \textbf{Implement the Reference Design:} Implement the original binary buddy allocator and verify its functionality to establish a foundation for further development.
    \item \textbf{Investigate Key Aspects:} Identify important aspects of using a memory allocator within a garbage collection context through research and experimentation with source code.
    \item \textbf{Make Modifications Based on Existing Research:} Modify the reference allocator based on prior research to enhance its performance.
    \item \textbf{Adapt for Garbage Collection:} Further modify the allocator for specific use cases within garbage collection.
    \item \textbf{Evaluate Performance:} Measure memory fragmentation and allocation speed for both the reference allocator and the modified versions.
\end{enumerate}

The initial phase involved implementing the original binary buddy allocator as envisioned by Knowlton \cite{buddy}. This implementation provided a deeper understanding of its functionality and established a solid foundation for subsequent development. The base allocator was tested using both unit tests and real-world applications. By exporting the allocator as a shared library within a wrapper, it was possible to test real-world programs by overriding the malloc/free operations using the \texttt{LD\_PRELOAD} environment variable.

Once the base allocator was implemented, phase two focused on understanding the nuances of memory allocation within the context of garbage collection and ZGC. This involved conducting a literature review, examining specific parts of the ZGC source code, and performing experiments. This phase was crucial for grasping the complexities of ZGC and its memory allocation mechanisms, identifying potential areas for modification, and determining which changes would have the greatest impact.

With a thorough understanding of the base allocator and ZGC, the third phase involved integrating modifications inspired by prior research. The most significant changes were applied, sometimes requiring the implementation of conflicting concepts, which were then compared and evaluated. The focus was on leveraging existing work to enhance overall performance, rather than to start from scratch. These adjustments were not exclusively tailored for garbage collection but aimed to improve general performance or produce positive results when used in garbage collection processes.

In the fourth phase, distinct improvements were implemented that were specifically relevant to the allocator's use in a garbage collection environment. These modifications were derived from insights and knowledge acquired in the preceding phases, rather than based on existing research. The goal was to introduce features tailored for specific scenarios within ZGC without compromising performance in other aspects of ZGC or elsewhere.

The implementation process was incremental, with each minor modification being tested and validated before proceeding to the next step. As the allocator evolved, the number of unit tests increased, which ensured comprehensive test coverage of its capabilities. Additionally, the allocator was tested with preexisting programs after each modification to ensure stability and performance.

The final phase involved evaluating the reference allocator and the modifications. Several key metrics were considered during the evaluation, including fragmentation, memory usage, and allocation speed. These metrics were essential for assessing the allocator's performance and comparing different versions. Certain metrics may hold greater significance in specific scenarios, while others may have different requirements. A detailed explanation of the evaluation process can be found in Section~\ref{sec:evaluation}.

%%% Local Variables:
%%% mode: latex
%%% TeX-master: "main"
%%% End:


\section{Evaluation Methodology}
\label{sec:evaluation}
Performance, memory usage, and fragmentation of the allocators are critical metrics evaluated in this study. These are important factors for a memory allocator, and improvements in one often affect others. Performance is measured by the time required for allocation and deallocation requests. Memory usage refers to the additional memory utilized by the allocator for its operations, and fragmentation refers to the inefficiencies in memory utilization resulting from the allocation strategy.

As previously stated, a significant constraint of this project is the non-integration of the allocator into ZGC, despite this being its intended purpose. This integration would provide the simplest and most intuitive method of testing the allocator. The challenge of not doing this is to identify other benchmarks that can indicate the allocator's performance within ZGC.

\subsection{Allocator Configurations}
The three versions of the buddy allocator: the binary buddy allocator, the binary tree allocator, and the iBuddy allocator; are tested and compared against each other. They all share the same base configuration of block sizes and regions, with a minimum block size of 16 bytes, a maximum block size of 256 KiB, and 8 total regions.

\subsection{Performance}
Performance is influenced by both the type of allocator used and the context of its use. Different allocators perform differently depending on the task; therefore, their performance varies between different programs. Instead of relying on a single program to compare all allocators, primitive measurements will be used to show how the allocators perform in allocating/deallocating various sizes and numbers of blocks.

\subsubsection{Measuring Performance}
Observing the time required for a single allocation/deallocation for each block size provides insight into the performance of each allocator. Additionally, timing the allocation of a constant memory size using various block sizes demonstrates how repeated allocation/deallocations perform.

Measurements will be taken for every possible block size allocation ranging from $2^4$ to $2^{18}$. These figures match the potential block sizes in a ZGC small page. To minimize noise, the average of $100 000$ allocations will be calculated for each size and each allocator. The duration will be recorded before and after the allocation call using the POSIX function \texttt{clock\_gettime()} with the \texttt{CLOCK\_MONOTONIC\_RAW} clock.

Another test involves allocating a larger memory block. This test will fill a contiguous 2 MiB memory block for every possible block size ranging from $2^4$ to $2^{18}$, matching ZGC's allocation sizes and page size. For each allocator and block size, the memory will be filled $10 000$ times. The duration will be recorded using the POSIX function \texttt{clock\_gettime()} with the \texttt{CLOCK\_MONOTONIC\_RAW} clock, starting prior to the initial allocation and ending after the final one.

These tests show different performance aspects of the different allocators for various allocation sizes, which can be used to draw several conclusions. To estimate the performance of a particular program, one would examine their allocation distribution and compare it with the test results to determine the most suitable allocator for that specific application.

\subsubsection{System Specifications}
Performance evaluations are performed using two different machines, detailed in Table~\ref{table:performancespecs}. Each memory allocator is compiled using GCC 11.4.0, adhering to the C++14 standard (\texttt{-std=c++14}) and using optimization level two (\texttt{-O2}).

\begin{table}[h]
    \begin{tabular}{lll}
        \textbf{Configuration}    & \textbf{Machine A}                                       & \textbf{Machine B}   \\ \hline
        CPU                       & Intel® Core™ i7-1270P                                    & AMD Opteron™ 6282 SE \\ \hline
        Sockets / Cores / Threads & 1 / 4P 8E / 16                                           & 2 / 16 / 32          \\ \hline
        Frequency (Base / Turbo)  & 2.2 GHz / 4.8 GHz                                        & 2.6 GHz / 3.3 GHz    \\ \hline
        L1 Cache                  & 448 KiB                                                  & 512 KiB              \\ \hline
        L2 Cache                  & 9 MiB                                                    & 32 MiB               \\ \hline
        L3 Cache                  & 18 MiB                                                   & 24 MiB               \\ \hline
        Memory                    & 16 GiB                                                   & 126 GiB              \\ \hline
        OS                        & \multicolumn{2}{l}{Ubuntu 22.04.4 LTS (Jammy Jellyfish)}                        \\ \hline
        Kernel                    & 6.5.0-17-generic                                         & 5.15.0-101-generic   \\
    \end{tabular}
    \centering
    \caption{Machines used for performance benchmarks.}
    \label{table:performancespecs}
\end{table}

\subsection{Fragmentation} \label{sec:frageval}
\subsubsection{Internal Fragmentation}
Internal fragmentation occurs when allocation sizes are rounded to the nearest power of two to align with block sizes, and is identical in all the allocators. This results in unused memory for every allocation that is not a perfect power of two. To quantify this, two counters are used: one for the requested allocation size and another for the total allocated size. Each allocation operation increments these counters, while deallocation operations decrement them. At any point during program execution, these two can be compared to measure the amount of internal fragmentation. The percentage of internal fragmentation, or wasted memory, can be calculated as

$\text{Internal fragmentation} = \frac{\text{Used memory - Requested memory}}{\text{Used memory}}$

\subsubsection{External Fragmentation}
External fragmentation occurs when allocated blocks are placed in memory in a way that inhibits larger allocations. Even if the total space available is more than a requested size, there may be no single contiguous block large enough to satisfy that request. The different allocators have different policies of where blocks are placed, so fragmentation will differ between them.

\subsubsection{Measuring Fragmentation}
Fragmentation is highly dependent on a specific pattern of allocation and deallocation. A program that consecutively only allocates powers of two will experience no internal or external fragmentation. However, this scenario is unrealistic, as programs typically mix allocations and deallocations across a broad range of sizes. To illustrate potential fragmentation levels in a program, we perform a simulation that uses the modified allocators in a way that creates high fragmentation. This simulation provides a rough estimate of the maximum fragmentation that could arise from typical use of the allocator.

This test uses two modules: one for creating a distribution of allocation sizes and another for producing a sequence of allocations and deallocations based on that distribution. Using these, it is possible to measure both internal and external fragmentation at any point within the sequence of allocations.

\newpage
The chosen allocation distribution is based on the observation that most allocations in Java programs are small, with a few large allocations and occasionally very large ones. Samples are drawn from a Poisson distribution with a mean of $\lambda = 6$. The samples are converted to allocation sizes by using them as exponents with base 2, and a variance of $\pm 50\%$ is applied to the converted samples to introduce spread.

To make the evaluation targeted towards ZGC, the values are aligned to $8$, and any values smaller than $2^4$ or larger than $2^{18}$ are excluded to adhere to the limited allocation size range for small pages. This results in a distribution centered on $2^6$, with most values clustering near this point and a gradual but steep decrease in the frequency of higher values.

The sequence of allocations and deallocations is created to introduce significant fragmentation without using a deliberate pattern. First, a sequence based on the allocation size distribution is generated. Following this sequence, allocations are made until a failure occurs. Subsequently, half of the allocations by size are randomly deallocated, resulting in fragmentation throughout the entire memory. This process is iterated numerous times to ensure extensive fragmentation.

External fragmentation is measured each time an allocation attempt is unsuccessful, providing insight into the allocator’s ability to manage fragmented memory. Internal fragmentation is measured simultaneously, capturing the impact of random size variations on memory utilization. This approach ensures a thorough evaluation of the allocator's performance under conditions of high fragmentation.

%%% Local Variables:
%%% mode: latex
%%% TeX-master: "main"
%%% End:


\section{Results}
\label{sec:results}

\subsection{Page Overhead}
The design of the buddy allocator requires storing additional metadata separate from the data on the page. This metadata, which relates to the status of possible blocks, incurs a fixed overhead for each page allocated, rather than increasing with each allocation. The size of the metadata is influenced by the number of blocks rather than their size. Consequently, larger block sizes lead to a minimal overhead percentage, whereas smaller block sizes result in significantly higher proportional overhead.

The basic binary buddy allocator was the most efficient in terms of memory usage, requiring the least data. The state of each potential block must be stored, which can be represented by a single bit. This data is only used during deallocation, so the states of two buddy blocks can be merged into one bit using XOR operations, effectively halving the memory requirement. In the ZGC configuration, this resulted in a total overhead of $16$ KiB, or $0.78$\%.

The iBuddy allocator also requires storing each block's state, but the same optimizations are not possible due to its inverse nature. Each block needs to set and read its state individually, as operations can be executed on multiple blocks at once. In the ZGC configuration, this led to an overall overhead of $32$ KiB, or $1.56$\%.

In contrast, the binary tree allocator requires additional information to be stored. Storing only the state of each block is insufficient, as each block needs to maintain a record of the highest level accessible below it. However, significant optimizations were applied to the lower layers, as described in Section~\ref{sec:adaptationsmetadata}, reducing the total overhead to $57$ KiB, or $2.7$\%.

% \vspace{-0.2cm}
\subsection{Performance}
The results of the single allocation benchmark are presented in Figure~\ref{fig:allocbenchmark}. Figures \ref{fig:allocA} and \ref{fig:allocB} show the results for machines A and B, respectively. The performance of the allocators was similar across both machines, differing by a constant speed factor.

Both the binary buddy and binary tree allocators demonstrated quicker allocation of larger blocks, with the latter showing superior performance for smaller block sizes. The iBuddy allocator was faster at allocating smaller blocks but had reduced speed for larger blocks. Utilizing the lazy layer significantly improved performance across all block sizes, with the greatest impact at smaller sizes.

\begin{figure}[h]
  \centering
  \begin{subfigure}{\textwidth}
    \centering
    \captionsetup{justification=centering}
    \includesvg[width=1.02\linewidth]{figures/alloc_laptop.svg}
    \caption{Allocation benchmark results from machine A.}
    \label{fig:allocA}
  \end{subfigure}
  \vspace{-0.5cm}
  \rule{\textwidth}{0.1pt}
  \begin{subfigure}{\textwidth}
    \centering
    \captionsetup{justification=centering}
    \includesvg[width=1.02\linewidth]{figures/alloc_server.svg}
    \caption{Allocation benchmark results from machine B.}
    \label{fig:allocB}
  \end{subfigure}
  \caption{Performance of individual allocations across various block sizes and allocators. The benchmark is run on both machines, and each bar displays the mean results from 100 000 iterations.}
  \label{fig:allocbenchmark}
\end{figure}

\FloatBarrier

The results of the single deallocation benchmark are presented in Figure~\ref{fig:deallocbenchmark}. Figures \ref{fig:deallocA} and \ref{fig:deallocB} show the results for machines A and B, respectively. The performance of the allocators was similar across both machines, differing by a constant speed factor.

Both the binary buddy and binary tree allocators demonstrated quicker deallocation of larger blocks, with the latter showing superior performance for smaller block sizes, although less than during allocation. The iBuddy allocator was faster at deallocating smaller blocks but had reduced speed for larger blocks. Utilizing the lazy layer significantly improved performance across all block sizes, with the greatest impact at smaller sizes.

\begin{figure}[h]
  \centering
  \begin{subfigure}{\textwidth}
    \centering
    \captionsetup{justification=centering}
    \includesvg[width=1.02\linewidth]{figures/dealloc_laptop.svg}
    \caption{Deallocation benchmark results from machine A}
    \label{fig:deallocA}
  \end{subfigure}
  \rule{\textwidth}{0.1pt}
  \begin{subfigure}{\textwidth}
    \centering
    \captionsetup{justification=centering}
    \includesvg[width=1.02\linewidth]{figures/dealloc_server.svg}
    \caption{Deallocation benchmark results from machine B}
    \label{fig:deallocB}
  \end{subfigure}
  \caption{Performance of individual allocations across various block sizes and allocators. The benchmark is run on both machines, and each bar displays the mean results from 100 000 iterations.}
  \label{fig:deallocbenchmark}
\end{figure}

\FloatBarrier

The results of the page-fill benchmark are presented in Figure~\ref{fig:allocpage}. Figures \ref{fig:allocpageA} and \ref{fig:allocpageB} show the results for machines A and B, respectively. The performance of the allocators was similar across both machines, differing by a constant speed factor.

The results illustrate how the characteristics of each allocator manifested when making multiple repeat allocations. The binary buddy and binary tree allocators scale linearly with block size due to their improved allocation speed for larger blocks. The binary tree allocator was faster, as seen by it being lower in the graph than the binary buddy allocator. Conversely, the iBuddy allocator exhibited inverse scaling, being faster for small blocks but flattening out for larger block sizes.

\begin{figure}[h]
  \centering
  \begin{subfigure}{0.496\textwidth}
    \centering
    \captionsetup{justification=centering}
    \includesvg[width=1\linewidth]{figures/page_laptop.svg}
    \caption{Allocation results from machine A}
    \label{fig:allocpageA}
  \end{subfigure}
  \begin{subfigure}{0.496\textwidth}
    \centering
    \captionsetup{justification=centering}
    \includesvg[width=1\linewidth]{figures/page_server.svg}
    \caption{Allocation results from machine B}
    \label{fig:allocpageB}
  \end{subfigure}
  \caption{Time taken to allocate 2 MiB across various block sizes and allocator versions. The benchmark is run on both machines, and each point displays the mean results from 1 000 iterations.}
  \label{fig:allocpage}
\end{figure}

\FloatBarrier
\subsection{Fragmentation}
\subsubsection{Internal Fragmentation}
Internal fragmentation remains consistent across all allocator versions, as they round block sizes equally and used the same allocation pattern. Table~\ref{table:fraginternal} presents measures that quantify the internal fragmentation experienced by the allocators. Although fragmentation was considerable, the standard deviation is low, indicating consistent fragmentation with a low spread. The chosen allocation pattern greatly influenced fragmentation levels, causing nearly all allocations to require rounding up.

\begin{table}[h]
  \begin{tabular}{|l|l|}
    \hline
    \textbf{Measure}    & \textbf{Value} \\ \hline
    Minimum:            & 24.4\%         \\ \hline
    Maximum:            & 40.9\%         \\ \hline
    Mean:               & 32.0\%         \\ \hline
    Median:             & 32.3\%         \\ \hline
    Standard Deviation: & 2.1\%          \\ \hline
  \end{tabular}
  \centering
  \caption{Measurements of internal fragmentation based on 1,000 observations. All allocator versions experience the same internal fragmentation.}
  \label{table:fraginternal}
\end{table}

\subsubsection{External Fragmentation}
External fragmentation manifested itself differently in each allocator version due to their different strategies for placing allocated blocks. Figures \ref{fig:fragextbinary}, \ref{fig:fragextbt}, and \ref{fig:fragextibuddy} illustrate the distribution of free blocks in terms of number and total size for the binary buddy, binary tree, and iBuddy allocators, respectively.

The binary buddy allocator showed the largest number of free blocks around $2^7$, but these constituted a minor portion of the overall free space. Most of the space was occupied by blocks around $2^{14}$, suggesting that allocations of this size were possible, and most unsuccessful allocations exceeded this size.

\begin{figure}[h]
  \centering
  \includesvg[width=1\textwidth]{figures/frag_binary.svg}
  \caption{Mean external fragmentation over 1 000 observations of the binary buddy allocator when allocating and deallocating randomly.}
  \label{fig:fragextbinary}
\end{figure}

The binary tree allocator was similar to the binary buddy allocator, with the largest number of blocks around $2^7$. Although it had more smaller-sized blocks, these made up only a small fraction of the total free space. Most of the space was occupied by blocks around $2^{13}$, indicating that the binary tree allocator was more efficient in using memory than the binary buddy allocator.

\begin{figure}[h]
  \centering
  \includesvg[width=1\textwidth]{figures/frag_bt.svg}
  \caption{Mean external fragmentation over 1 000 observations of the binary tree allocator when allocating and deallocating randomly.}
  \label{fig:fragextbt}
\end{figure}

The iBuddy allocator stands out due to its aggressive splitting policy. The most common size of free blocks was $2^6$, which also constituted a substantial portion of the total free space. The largest total space was found around $2^{13}$, although it was more evenly distributed between sizes. This concentration at $2^6$ resulted from most allocations exceeding this size, which led to blocks being split until this size. The overall free space was greater than that of the other two allocators, suggesting less efficiency in utilizing memory.

\begin{figure}[h]
  \centering
  \includesvg[width=1\textwidth]{figures/frag_ibuddy.svg}
  \caption{Mean external fragmentation over 1 000 observations of the iBuddy allocator when allocating and deallocating randomly.}
  \label{fig:fragextibuddy}
\end{figure}

%%% Local Variables:
%%% mode: latex
%%% TeX-master: "main"
%%% End:


\section{Discussion}
\label{sec:discussion}

The evaluation of the adapted allocators presents considerable challenges. Because these allocators are not integrated into ZGC, definitive conclusions regarding their performance are difficult to reach. Only proxy benchmarks are possible, serving as indicators of potential performance. Isolating the allocators for testing removes the influence of program logic, whether from ZGC or other user-defined logic. This isolation can result in, for example, cache-locality effects that do not accurately reflect real-world scenarios. However, these assessments allow us to compare the allocators on a uniformly defined basis and to discuss their comparative effectiveness.

Performance and external fragmentation benchmarks demonstrate that the binary tree allocator consistently outperforms the binary buddy allocator but at the cost of increased memory overhead. Furthermore, the iBuddy allocator exhibits significant differences in allocation performance and external fragmentation distribution compared to the other two allocators. The choice of allocator for a particular scenario should depend on the most critical performance metric for that use case.

Internal fragmentation is the most significant issue faced by all allocators. The requirement to round up block sizes leads to considerable waste, especially with larger allocations. Since most allocations do not align with a perfect power of two, fragmentation occurs with each allocation, diminishing the efficiency of any buddy allocator in utilizing memory space. In the context of ZGC, this problem can be mitigated by strategically grouping object allocations to minimize internal fragmentation. Further discussion of this strategy is found in Section~\ref{sec:futureworkZ} as future work.

External fragmentation can only be measured empirically, which is why the allocation/deallocation pattern is chosen to induce significant fragmentation without maximizing it. Under this pattern, all allocator versions experience high external fragmentation, particularly at larger block sizes, where most of the total space is located. This suggests that failures predominantly occur with very large allocation requests and that a wide range of allocation sizes presents a major weakness, increasing the likelihood of failures for large requests. Limiting the range of sizes would likely decrease external fragmentation. Consequently, within ZGC, it should be presumed that additional allocatable space exists even if an allocation attempt fails. Assuming that memory is fully utilized could lead to considerable memory wastage.

The unique behavior of the iBuddy allocator is evident in the benchmarks. Notably, its performance for large block allocations is significantly worse comparatively to small allocations in the other two allocators. Its only advantage is a slight increase in speed at the smallest allocation sizes. However, its strategy of aggressively splitting blocks quickly inhibits the allocation of larger blocks. Therefore, the iBuddy allocator is only feasible when nearly all allocations are among the smallest sizes. If a considerable number of allocations exceed this size, there will be a notable decrease in both performance and usable memory.

Considering all factors, the binary tree allocator emerges as the most promising option for implementation in ZGC among the three discussed. It consistently outperforms the binary buddy allocator in all benchmarks, while the iBuddy allocator is impractical due to the reasons discussed. Since memory consumption is not a critical concern, the additional memory expense of the binary tree allocator is justified by its speed improvement. Although the binary tree allocator is slower at smaller allocation sizes, this drawback can be mitigated by using the lazy layer, ensuring optimal performance across all allocation sizes. This configuration is deemed optimal for integrating a buddy allocator within ZGC.

However, the intrinsic weaknesses of buddy allocators still remain. Internal fragmentation will be significant, even if it is possible to reduce it. Blocks still need to be aligned to their size, which could become problematic if previous memory is awkwardly placed. In addition, the cost of frequent allocator initialization could become non-negligible. These issues should be taken into account when integrating with ZGC, as they could affect both performance and memory efficiency. Future work should focus on strategies to mitigate these weaknesses and explore alternative allocation algorithms that may offer improved performance and reduced fragmentation in the context of ZGC.

%%% Local Variables:
%%% mode: latex
%%% TeX-master: "main"
%%% End:


\section{Conclusions}
\label{sec:conclusions}
This thesis has explored the adaptation and evaluation of different buddy allocators for integration within the Z Garbage Collector (ZGC). Through analysis and benchmarking, the binary tree allocator emerged as the most promising candidate, consistently outperforming the traditional binary buddy allocator in all key benchmarks. Although the iBuddy allocator demonstrated faster allocation speeds for smaller blocks, its poor performance with larger block allocations and significant internal fragmentation makes it impractical for this context.

By tailoring the allocator to the specific needs of ZGC, several changes have been implemented to enhance efficiency and introduce additional features. Using already-available data allowed for further optimizations. Notably, a novel approach to initializing and employing the allocator was implemented: it begins fully occupied, with the user identifying which sections are empty and available for allocation. These modifications are designed to facilitate the seamless integration of the allocator into ZGC in the future.

To fully understand how the allocator would perform in ZGC, additional research is needed. So far, only proxy benchmarks have been utilized, showing promise for potential real-world applications. Therefore, future efforts should focus on developing strategies to mitigate the fundamental shortcomings of buddy allocators in terms of internal fragmentation and alignment. Additionally, integrating the allocator within ZGC would enable extensive testing and would be crucial in refining the allocator's performance.

%%% Local Variables:
%%% mode: latex
%%% TeX-master: "main"
%%% End:


%%%% Referenser - SE OCKÅ APPENDIX

% Use one of these:
%   IEEEtranS gives numbered references like [42] sorted by author,
%   IEEEtranSA gives ``alpha''-style references like [Lam81] (also sorted by author)
\bibliographystyle{IEEEtranS}
%\bibliographystyle{IEEEtranSA}

% Here comes the bibliography/references.
% För att göra inställningar för IEEEtranS/SA kan man använda ett speciellt bibtex-entry @IEEEtranBSTCTL,
% se IEEEtran/bibtex/IEEEtran_bst_HOWTO.pdf, avsnitt VII, eller sista biten av IEEEtran/bibtex/IEEEexample.bib.
\bibliography{bibconfig,refs}
%\bibliography{refs}

\newpage
\appendix %%%% markerar att resten är appendixar
%%%% I er egen version, ta bort allt nedan (utom \end{document})
\input{text/appendix}

% Om ni har ett index
\makeatletter
\renewenvironment{theindex}
{\if@twocolumn
    \@restonecolfalse
  \else
    \@restonecoltrue
  \fi
  \twocolumn[\section{\indexname}]%
  \@mkboth{\MakeUppercase\indexname}%
  {\MakeUppercase\indexname}%
  \thispagestyle{plain}\parindent\z@
  \parskip\z@ \@plus .3\p@\relax
  \columnseprule \z@
  \columnsep 35\p@
  \let\item\@idxitem}
{\if@restonecol\onecolumn\else\clearpage\fi}
\makeatother
\printindex
\end{document}
